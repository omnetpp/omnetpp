


\subsection{Exposing statistics as signals}

motivation: we don't know in advance what, and with what level of detail we want to collect.

- level of detail: record all values as vector; record the mean/stddev/min/max/avg;
  or record a histogram as well;
- we you may want to process the results before recording them, i.e. record the sum,
  the time average, the count, a smoothed value etc).
- you may want aggregate statistics, e.g. packet drop count for the whole network
- or derived statistics, e.g. drop percentage (drop count/total pks)

These are all possible with signals.

@namespace(bubu);
simple Queue {
    @class(SimpleQueue);
    @signal[queueLength](type=double;interpolation-mode=...);
}

**.queue.queueLength.on-signal = vector(name="queue length", interval=0..500s), scalar(name=..,operation=count), listener(class="XXXListener",....)
**.queue.queueLength*.on-signal = listener(class="SignalJoiner";signals="length,limit,drop";output="")

Queuenet.e.queueLength.on-signal = listener(class="SignalJoiner";signals="length,limit,drop";output="")

**.queue.queueLength*.on-signal = listener:INET::Aggregate(...),
**.queue.queueLength*.on-signal = listener=INET::Aggregate(...),




%%% Local Variables:
%%% mode: latex
%%% TeX-master: "usman"
%%% End:
