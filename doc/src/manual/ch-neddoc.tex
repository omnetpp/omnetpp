\chapter{Documenting NED and Messages}
\label{cha:neddoc}

\section{Overview}
\label{sec:neddoc:overview}

{\opp} provides a tool which can generate HTML documentation from NED files
and message definitions. Like Javadoc and Doxygen, the NED documentation tool
makes use of source code comments. The generated HTML documentation
lists all modules, channels, messages, etc., and presents their details including
description, gates, parameters, assignable submodule parameters, and
syntax-highlighted source code. The documentation also includes clickable
network diagrams (exported from the graphical editor) and usage diagrams as
well as inheritance diagrams.

The documentation tool integrates with Doxygen, meaning that it can
hyperlink simple modules and message classes to their C++ implementation
classes in the Doxygen documentation. If you also generate the C++
documentation with some Doxygen features turned on (such as
\textit{inline-sources} and \textit{referenced-by-relation}, combined with
\textit{extract-all}, \textit{extract-private} and
\textit{extract-static}), the result is an easily browsable and very
informative presentation of the source code. Of course, one still has to
write documentation comments in the code.

In the 4.0 version, the documentation tool is part of the Eclipse-based
simulation IDE.


\section{Documentation Comments}
\label{sec:neddoc:documentation-comments}

Documentation is embedded in normal comments. All \texttt{//} comments
that are in the ``right place'' (from the documentation tool's
point of view) will be included in the generated documentation.
  \footnote{In contrast, Javadoc and Doxygen use special comments (those
     beginning with \texttt{/**}, \texttt{///}, \texttt{//<} or a similar
     marker) to distinguish documentation from ``normal'' comments in the
     source code. In {\opp} there is no need for that: NED and the message
     syntax is so compact that practically all comments one would want to write
     in them can serve documentation purposes.}

Example:

\begin{ned}
//
// An ad-hoc traffic generator to test the Ethernet models.
//
simple Gen
{
    parameters:
        string destAddress;  // destination MAC address
        int protocolId;      // value for SSAP/DSAP in Ethernet frame
        double waitMean @unit(s); // mean for exponential interarrival times
    gates:
        output out;  // to Ethernet LLC
}
\end{ned}

You can also place comments above parameters and gates. This is useful
if they need long explanations. Example:

\begin{ned}
//
// Deletes packets and optionally keeps statistics.
//
simple Sink
{
    parameters:
        // You can turn statistics generation on and off. This is
        // a very long comment because it has to be described what
        // statistics are collected (or not).
        bool collectStatistics = default(true);
    gates:
        input in;
}
\end{ned}

\subsection{Private Comments}
\label{sec:neddoc:private-comments}

If you want a comment line \textit{not} to appear in the documentation,
begin it with \texttt{//\#}. Those lines will be ignored by the
documentation tool, and can be used to make ``private'' comments
like \texttt{FIXME} or \texttt{TODO}, or to comment out unused code.

\begin{ned}
//
// An ad-hoc traffic generator to test the Ethernet models.
//# TODO above description needs to be refined
//
simple Gen
{
    parameters:
        string destAddress;  // destination MAC address
        int protocolId;      // value for SSAP/DSAP in Ethernet frame
        //# double burstiness;  -- not yet supported
        double waitMean @unit(s); // mean for exponential interarrival times
    gates:
        output out;  // to Ethernet LLC
}
\end{ned}


\subsection{More on Comment Placement}
\label{sec:neddoc:comment-placement}

Comments should be written where the tool will find them.
This is a) immediately above the documented item, or b) after the
documented item, on the same line.

In the former case, make sure there is no blank line left
between the comment and the documented item. Blank lines
detach the comment from the documented item.

Example:
\begin{ned}
// This is wrong! Because of the blank line, this comment is not
// associated with the following simple module!

simple Gen
{
    ...
}
\end{ned}

Do not try to comment groups of parameters together. The result
will be awkward.

\section{Referring to Other NED and Message Types}
\label{sec:neddoc:referring-to-other-ned-and-message-types}

You can reference other NED and message types by name in comments. There
are two styles in which references can be written: automatic linking and
tilde linking. The same style must be following throughout the whole
project, and the correct one must be selected in the documentation
generator tool when it is run.

\subsection{Automatic Linking}
\label{sec:neddoc:automatic-linking}

In the automatic linking style, words that match existing NED of message
types are hyperlinked automatically. It is usually enough to write the
simple name of the type (e.g. \ttt{TCP}), you don't need to spell out the
fully qualified type (\ttt{inet.transport.tcp.TCP}), although you can.

Automatic hyperlinking is sometimes overly agressive. For example, when you
write \ttt{IP address} in a comment and an \ttt{IP} module exists
in the project, it will create a hyperlink to the module, which is probably
not what you want. You can prevent hyperlinking of a word by inserting a
backslash in front it: \ttt{{\textbackslash}IP address}. The backslash will
not appear in the HTML output. The \ttt{<nohtml>} tag will also prevent
hyperlinking words in the enclosed text: \ttt{<nohtml>IP address</nohtml>}.
On the other hand, if you deliberately want to print a backslash immediately
in front of a word (e.g. output \textit{``use {\textbackslash}t to print a Tab''}),
use either two backslashes (\ttt{use {\textbackslash}{\textbackslash}t...}) or the
\ttt{<nohtml>} tag (\ttt{<nohtml>use {\textbackslash}t...</nohtml>}).
Backslashes in other contexts (i.e. when not in front of a word) do not have
a special meaning, and are preserved in the output.

The detailed rules:

\begin{enumerate}
  \item Words matching a type name are automatically hyperlinked
  \item A backslash immediately followed by an identifier (i.e. letter or underscore)
        prevents hyperlinking, and the backslash is removed from the output
  \item A double backslash followed by an identifier produces a single backslash,
        plus the potentially hyperlinked identifier
  \item Backslashes in any other contexts are not interpreted, and preserved in the output
  \item Tildes are not interpreted, and are preserved in the output
  \item Inside \ttt{<nohtml>}, no backslash processing or hyperlinking takes place
\end{enumerate}

\subsection{Tilde Linking}
\label{sec:neddoc:tilde-linking}

In the tilde style, only words that are explicitly marked with a tilde are
subject to hyperlinking: \ttt{{\textasciitilde}TCP},
\ttt{{\textasciitilde}inet.transport.tcp.TCP}.

To produce a literal tilde followed by an identifier in the output (for example,
to output \textit{``the {\textasciitilde}TCP() destructor''}), you need to
double the tilde character: \ttt{the {\textasciitilde}{\textasciitilde}TCP() destructor}.

The detailed rules:

\begin{enumerate}
  \item Words matching a type name are \textit{not} hyperlinked automatically
  \item A tilde immediately followed by an identifier (i.e. letter or underscore)
        will be hyperlinked, and the tilde is removed from the output. It is
        considered an error if there is no type with that name.
  \item A double tilde followed by an identifier produces a single tilde plus the identifier
  \item Tildes in any other contexts are not interpreted, and preserved in the output
  \item Backslashes are not interpreted, and are preserved in the output
  \item Inside \ttt{<nohtml>}, no tilde processing or hyperlinking takes place
\end{enumerate}

\section{Text Layout and Formatting}
\label{sec:neddoc:text-layout-and-formatting}

\subsection{Paragraphs and Lists}
\label{sec:neddoc:paragraphs-and-lists}

If you write longer descriptions, you will need text formatting capabilities.
Text formatting works like in Javadoc or Doxygen -- you can break up the
text into paragraphs and create bulleted/numbered lists without
special commands, and use HTML for more fancy formatting.

Paragraphs are separated by empty lines, like in LaTeX or Doxygen.
Lines beginning with `\ttt{-}' will be turned into bulleted lists,
and lines beginning with `\ttt{-\#}' into numbered lists.

Example:

\begin{ned}
//
// Ethernet MAC layer. MAC performs transmission and reception of frames.
//
// Processing of frames received from higher layers:
// - sends out frame to the network
// - no encapsulation of frames -- this is done by higher layers.
// - can send PAUSE message if requested by higher layers (PAUSE protocol,
//   used in switches). PAUSE is not implemented yet.
//
// Supported frame types:
// -# IEEE 802.3
// -# Ethernet-II
//
\end{ned}


\subsection{Special Tags}
\label{sec:neddoc:special-tags}

The documentation tool understands the following tags and will render them accordingly:
\ttt{@author}, \ttt{@date}, \ttt{@todo}, \ttt{@bug}, \ttt{@see}, \ttt{@since},
\ttt{@warning}, \ttt{@version}. Example usage:

\begin{ned}
//
// @author Jack Foo
// @date 2005-02-11
//
\end{ned}


\subsection{Text Formatting Using HTML}
\label{sec:neddoc:text-formatting-using-html}

Common HTML tags are understood as formatting commands.
The most useful tags are: \ttt{<i>..</i>} (italic),
\ttt{<b>..</b>} (bold), \ttt{<tt>..</tt>} (typewriter font),
\ttt{<sub>..</sub>} (subscript), \ttt{<sup>..</sup>} (superscript),
\ttt{<br>} (line break), \ttt{<h3>} (heading),
\ttt{<pre>..</pre>} (preformatted text) and \ttt{<a href=..>..</a>} (link),
as well as a few other tags used for table creation (see below).
For example, \texttt{<i>Hello</i>} will be rendered as ``\textit{Hello}''
(using an italic font).

The complete list of HTML tags interpreted by the documentation tool are:
\texttt{<a>}, \texttt{<b>}, \texttt{<body>}, \texttt{<br>}, \texttt{<center>},
\texttt{<caption>}, \texttt{<code>}, \texttt{<dd>}, \texttt{<dfn>}, \texttt{<dl>},
\texttt{<dt>}, \texttt{<em>}, \texttt{<form>}, \texttt{<font>}, \texttt{<hr>},
\texttt{<h1>}, \texttt{<h2>}, \texttt{<h3>}, \texttt{<i>}, \texttt{<input>}, \texttt{<img>},
\texttt{<li>}, \texttt{<meta>}, \texttt{<multicol>}, \texttt{<ol>}, \texttt{<p>}, \texttt{<small>},
\texttt{<span>}, \texttt{<strong>},
\texttt{<sub>}, \texttt{<sup>}, \texttt{<table>}, \texttt{<td>}, \texttt{<th>}, \texttt{<tr>},
\texttt{<tt>}, \texttt{<kbd>}, \texttt{<ul>}, \texttt{<var>}.

Any tags not in the above list will not be interpreted as formatting commands
but will be printed verbatim -- for example, \texttt{<what>bar</what>}
will be rendered literally as ``<what>bar</what>'' (unlike HTML where
unknown tags are simply ignored, i.e. HTML would display ``bar'').

If you insert links to external pages (web sites), its useful to add
the \ttt{target="\_blank"} attribute to ensure pages come up in a new
browser window and not just in the current frame which looks awkward.
(Alternatively, you can use the \ttt{target="\_top"} attribute
which replaces all frames in the current browser).

Examples:

\begin{ned}
//
// For more info on Ethernet and other LAN standards, see the
// <a href="http://www.ieee802.org/" target="_blank">IEEE 802
// Committee's site</a>.
//
\end{ned}

You can also use the \ttt{<a href=..>} tag to create links within the page:

\begin{ned}
//
// See the <a href="#resources">resources</a> in this page.
// ...
// <a name="resources"><b>Resources</b></a>
// ...
//
\end{ned}

You can use the \texttt{<pre>..</pre>} HTML tag to insert source code examples
into the documentation. Line breaks and indentation will be preserved,
but HTML tags continue to be interpreted (or you can turn them off
with \texttt{<nohtml>}, see later).

Example:

\begin{ned}
// <pre>
// // my preferred way of indentation in C/C++ is this:
// <b>for</b> (<b>int</b> i = 0; i < 10; i++) {
//     printf(<i>"%d\n"</i>, i);
// }
// </pre>
\end{ned}

will be rendered as

\begin{Verbatim}[commandchars=\\\{\}]
// my preferred way of indentation in C/C++ is this:
\textbf{for} (\textbf{int} i = 0; i < 10; i++) \{
    printf(\textit{"%d{\textbackslash}n"}, i);
\}
\end{Verbatim}

HTML is also the way to create tables. The example below

\begin{ned}
//
// <table border="1">
//   <tr>  <th>#</th> <th>number</th> </tr>
//   <tr>  <td>1</td> <td>one</td>    </tr>
//   <tr>  <td>2</td> <td>two</td>    </tr>
//   <tr>  <td>3</td> <td>three</td>  </tr>
// </table>
//
\end{ned}

will be rendered approximately as:

\begin{longtable}{|l|l|}
\hline
\tabheadcol
\tbf{\#} & \tbf{number} \\\hline
1 & one \\\hline
2 & two \\\hline
3 & three \\\hline
\end{longtable}


\subsection{Escaping HTML Tags}
\label{sec:neddoc:escaping-html-tags}

Sometimes you may need to turn off interpreting HTML tags (\ttt{<i>}, \ttt{<b>}, etc.)
as formatting instructions, and rather you want them to appear as literal
\ttt{<i>}, \ttt{<b>} text in the documentation. You can achieve this via
surrounding the text with the \ttt{<nohtml>}...\ttt{</nohtml>} tag.
For example,

\begin{ned}
// Use the <nohtml><i></nohtml> tag (like <tt><nohtml><i>this</i></nohtml><tt>)
// to write in <i>italic</i>.
\end{ned}

will be rendered as ``Use the <i> tag (like \texttt{<i>this</i>}) to write
in \textit{italic}.''

\ttt{<nohtml>}...\ttt{</nohtml>} will also prevent \fprog{opp\_neddoc}
from hyperlinking words that are accidentally the same as an existing
module or message name. Prefixing the word with a backslash will achieve
the same. That is, either of the following will do:

\begin{ned}
// In <nohtml>IP</nohtml> networks, routing is...
\end{ned}

\begin{ned}
// In \IP networks, routing is...
\end{ned}

Both will prevent hyperlinking the word \textit{IP} if you happen to have
an \ttt{IP} module in the NED files.



\section{Customizing and Adding Pages}
\label{sec:neddoc:customizing-and-adding-pages}

\subsection{Adding a Custom Title Page}
\label{sec:neddoc:adding-custom-title-page}

The title page is the one that appears in the main frame after
opening the documentation in the browser. By default it contains
a boilerplate text with the generic title \textit{``{\opp} Model Documentation''}.
You probably want to customize that, and at least change the title
to the name of the documented simulation model.

You can supply your own version of the title page adding a \ttt{@titlepage}
directive to a file-level comment (a comment that appears at the top of
a NED file, but is separated from the first \ttt{import}, \ttt{channel},
\ttt{module}, etc. definition by at least one blank line).
In theory you can place your title page definition into
any NED or MSG file, but it is probably a good idea to create
a separate \ttt{package.ned} file for it.

The lines you write after the \ttt{@titlepage} line up to the next
\ttt{@page} line (see later) or the end of the comment will be used
as the title page.
You probably want to begin with a title because the documentation
tool doesn't add one (it lets you have full control over the
page contents). You can use the \ttt{<h1>..</h1>} HTML tag
to define a title.

Example:

\begin{ned}
//
// @titlepage
// <h1>Ethernet Model Documentation</h1>
//
// This documents the Ethernet model created by David Wu and refined by Andras
// Varga at CTIE, Monash University, Melbourne, Australia.
//
\end{ned}


\subsection{Adding Extra Pages}
\label{sec:neddoc:adding-extra-pages}

You can add new pages to the documentation in a similar way as customizing
the title page. The directive to be used is \ttt{@page}, and it can
appear in any file-level comment (see above).

The syntax of the \ttt{@page} directive is the following:

\begin{ned}
// @page filename.html, Title of the Page
\end{ned}

Choose a file name that doesn't collide with the files generated
by the documentation tool (such as \ttt{index.html}). If the file name
does not end in \ttt{.html} already, it will be appended.
The page title you supply will appear on the top of the page as well as
in the page index.

The lines after the \ttt{@page} line up to the next \ttt{@page} line
or the end of the comment will be used as the page body.
You don't need to add a title because the documentation tool
automatically adds one.

Example:
\begin{ned}
//
// @page structure.html, Directory Structure
//
// The model core model files and the examples have been placed
// into different directories. The <tt>examples/</tt> directory...
//
//
// @page examples.html, Examples
// ...
//
\end{ned}

You can create links to the generated pages using standard HTML,
using the \ttt{<a href="...">...</a>} tag. All HTML files are
placed in a single directory, so you don't have to worry about
specifying directories.

Example:
\begin{ned}
//
// @titlepage
// ...
// The structure of the model is described <a href="structure.html">here</a>.
//
\end{ned}


\subsection{Incorporating Externally Created Pages}
\label{sec:neddoc:externally-created-pages}

You may want to create pages outside the documentation tool
(e.g. using a HTML editor) and include them in the documentation.
This is possible, all you have to do is declare such pages with
the \ttt{@externalpage} directive in any of the NED files, and
they will be added to the page index. The pages can then be linked to
from other pages using the HTML \ttt{<a href="...">...</a>} tag.

The \ttt{@externalpage} directive is similar in syntax to \ttt{@page}:

\begin{ned}
// @externalpage filename.html, Title of the Page
\end{ned}

The documentation tool does not check if the page exists
or not. It is your responsibility to copy it manually into
the directory of the generated documentation, and to make
sure the hyperlink works.

\section{File Inclusion}
\label{sec:neddoc:file-inclusion}

You can include content into the documentation comment with the
\ttt{@include} directive. It expects the path of the file to be incuded
relative to the file that includes it.

The line of the \ttt{@include} directive will be replaced by the
content of the file. The lines of the included file do not need
to start with \texttt{//}, but otherwise they are processed in the same way
as the NED comments. They can include other files, but circular
includes are not allowed.

\begin{ned}
// ...
// @include ../copyright.txt
// ...
\end{ned}

