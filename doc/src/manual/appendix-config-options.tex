\appendixchapter{Configuration Options}
\label{cha:config-options}

\section{Configuration Options}
\label{sec:config-options:config-options}

%% To bring this chapter up to date, run opp_run -h latexconfig, and paste
%% the result. When committing, make sure no option gets lost (e.g parsim
%% or MPI-related options; opp_run can only print options from code that
%% enabled at compile time!)
%% ALSO TAKE CARE THAT THE SEPARATE section{Predefined Variables} DOES NOT DISAPPEAR!

This section lists all configuration options that are available in ini files.
A similar list can be obtained from any simulation executable by running it
with the \fopt{-h configdetails} option.

%% Change tolerance to give more freedom to the hyphenation/word wrapping algorithm
%% \tolerance 5000

% @BEGINFILE tools/config.txt
\begin{description}
\item[abstract] = \textit{<bool>}, default: \ttt{false}\\
    \textit{Per-simulation-run setting.}\\
    Declares whether this configuration is an abstract one. Abstract
    configurations serve as a base for other configurations, and are not meant
    to be run directly.
\item[allow-object-stealing-on-deletion] = \textit{<bool>}, default: \ttt{false}\\
    \textit{Per-simulation-run setting.}\\
    Setting it to true disables the "Context component is deleting an object it
    doesn't own" error message. This option exists primarily for backward
    compatibility with pre-6.0 versions that were more permissive during object
    deletion.
\item[**.bin-recording] = \textit{<bool>}, default: \ttt{true}\\
    \textit{Per-object setting for scalar results.}\\
    Whether the bins of the matching histogram object should be recorded,
    provided that recording of the histogram object itself is enabled
    (\ttt{**.{\allowbreak}<scalar-{\allowbreak}name>.{\allowbreak}scalar-{\allowbreak}recording={\allowbreak}true}).\\
    Usage:
    \ttt{<module-{\allowbreak}full-{\allowbreak}path>.{\allowbreak}<scalar-{\allowbreak}name>.{\allowbreak}bin-{\allowbreak}recording={\allowbreak}true/{\allowbreak}false}.
    To control histogram recording from a \ttt{@{\allowbreak}statistic}, use
    \ttt{<statistic-{\allowbreak}name>:{\allowbreak}histogram} for
    \ttt{<scalar-{\allowbreak}name>}.\\
    Example:
    \ttt{**.{\allowbreak}ping.{\allowbreak}round\-Trip\-Time:{\allowbreak}histogram.{\allowbreak}bin-{\allowbreak}recording={\allowbreak}false}
\item[check-signals] = \textit{<bool>}, default: \ttt{true}\\
    \textit{Per-simulation-run setting.}\\
    Controls whether the simulation kernel will validate signals emitted by
    modules and channels against signal declarations
    (\ttt{@{\allowbreak}signal} properties) in NED files. The default setting
    depends on the build type: \ttt{true} in DEBUG, and \ttt{false} in RELEASE
    mode.
\item[cmdenv-autoflush] = \textit{<bool>}, default: \ttt{false}\\
    \textit{Per-simulation-run setting.}\\
    Call \ttt{fflush(stdout)} after each event banner or status update; affects
    both express and normal mode. Turning on autoflush may have a performance
    penalty, but it can be useful with printf-style debugging for tracking down
    program crashes.
\item[cmdenv-config-name] = \textit{<string>}\\
    \textit{Global setting (applies to all simulation runs).}\\
    Specifies the name of the configuration to be run (for a value \ttt{Foo},
    section \ttt{[Config {\allowbreak}Foo]} will be used from the ini file).
    See also
    \ttt{cmdenv-{\allowbreak}runs-{\allowbreak}to-{\allowbreak}execute}. The
    \ttt{-{\allowbreak}c} command line option overrides this setting.
\item[cmdenv-event-banner-details] = \textit{<bool>}, default: \ttt{false}\\
    \textit{Per-simulation-run setting.}\\
    When
    \ttt{cmdenv-{\allowbreak}express-{\allowbreak}mode={\allowbreak}false}:
    print extra information after event banners.
\item[cmdenv-event-banners] = \textit{<bool>}, default: \ttt{true}\\
    \textit{Per-simulation-run setting.}\\
    When
    \ttt{cmdenv-{\allowbreak}express-{\allowbreak}mode={\allowbreak}false}:
    turns printing event banners on/off.
\item[cmdenv-express-mode] = \textit{<bool>}, default: \ttt{true}\\
    \textit{Per-simulation-run setting.}\\
    Selects normal (debug/trace) or express mode.
\item[cmdenv-extra-stack] = \textit{<double>}, unit=\ttt{B}, default: \ttt{8Ki\-B}\\
    \textit{Global setting (applies to all simulation runs).}\\
    Specifies the extra amount of stack that is reserved for each
    \ttt{activity()} simple module when the simulation is run under Cmdenv.
\item[cmdenv-fake-gui] = \textit{<bool>}, default: \ttt{false}\\
    \textit{Per-simulation-run setting.}\\
    Causes Cmdenv to lie to simulations that is a GUI (isGui()=true), and to
    periodically invoke refreshDisplay() during simulation execution.
\item[cmdenv-fake-gui-after-event-probability] = \textit{<double>}, default: \ttt{1}\\
    \textit{Per-simulation-run setting.}\\
    When \ttt{cmdenv-{\allowbreak}fake-{\allowbreak}gui={\allowbreak}true}: The
    probability with which \ttt{refresh\-Display()} is called after each event.
\item[cmdenv-fake-gui-before-event-probability] = \textit{<double>}, default: \ttt{1}\\
    \textit{Per-simulation-run setting.}\\
    When \ttt{cmdenv-{\allowbreak}fake-{\allowbreak}gui={\allowbreak}true}: The
    probability with which \ttt{refresh\-Display()} is called before each
    event.
\item[cmdenv-fake-gui-on-hold-numsteps] = \textit{<custom>}, default: \ttt{3}\\
    \textit{Per-simulation-run setting.}\\
    When \ttt{cmdenv-{\allowbreak}fake-{\allowbreak}gui={\allowbreak}true}: The
    number of times \ttt{refresh\-Display()} is called during a "hold" period
    (animation during which simulation time does not advance), provided a trial
    with
    \ttt{cmdenv-{\allowbreak}fake-{\allowbreak}gui-{\allowbreak}on-{\allowbreak}hold-{\allowbreak}probability}
    yielded success. This an expression which will be evaluated each time, so
    it can be random. Zero is also a valid value.
\item[cmdenv-fake-gui-on-hold-probability] = \textit{<double>}, default: \ttt{0.{\allowbreak}5}\\
    \textit{Per-simulation-run setting.}\\
    When \ttt{cmdenv-{\allowbreak}fake-{\allowbreak}gui={\allowbreak}true}: The
    probability with which \ttt{refresh\-Display()} is called (possibly
    multiple times, see
    \ttt{cmdenv-{\allowbreak}fake-{\allowbreak}gui-{\allowbreak}on-{\allowbreak}hold-{\allowbreak}numsteps})
    during a "hold" period (animation during which simulation time does not
    advance).
\item[cmdenv-fake-gui-on-simtime-numsteps] = \textit{<custom>}, default: \ttt{3}\\
    \textit{Per-simulation-run setting.}\\
    When \ttt{cmdenv-{\allowbreak}fake-{\allowbreak}gui={\allowbreak}true}: The
    number of times \ttt{refresh\-Display()} is called when simulation time
    advances from one simulation event to the next, provided a trial with
    \ttt{cmdenv-{\allowbreak}fake-{\allowbreak}gui-{\allowbreak}on-{\allowbreak}simtime-{\allowbreak}probability}
    yielded success. This an expression which will be evaluated each time, so
    it can be random. Zero is also a valid value.
\item[cmdenv-fake-gui-on-simtime-probability] = \textit{<double>}, default: \ttt{0.{\allowbreak}1}\\
    \textit{Per-simulation-run setting.}\\
    When \ttt{cmdenv-{\allowbreak}fake-{\allowbreak}gui={\allowbreak}true}: The
    probability with which \ttt{refresh\-Display()} is called (possibly
    multiple times, see
    \ttt{cmdenv-{\allowbreak}fake-{\allowbreak}gui-{\allowbreak}on-{\allowbreak}simtime-{\allowbreak}numsteps})
    when simulation time advances from one simulation event to the next.
\item[cmdenv-fake-gui-seed] = \textit{<int>}, default: \ttt{1}\\
    \textit{Per-simulation-run setting.}\\
    When \ttt{cmdenv-{\allowbreak}fake-{\allowbreak}gui={\allowbreak}true}: The
    seed for the RNG governing the operation of the fake GUI component. This is
    entirely independent of the RNGs used by the model.
\item[cmdenv-interactive] = \textit{<bool>}, default: \ttt{false}\\
    \textit{Per-simulation-run setting.}\\
    Defines what Cmdenv should do when the model contains unassigned
    parameters. In interactive mode, it asks the user. In non-interactive mode
    (which is more suitable for batch execution), Cmdenv stops with an error.
\item[cmdenv-log-cleanup] = \textit{<bool>}, default: \ttt{false}\\
    \textit{Per-simulation-run setting.}\\
    Specifies whether to enable logging during network cleanup.
\item[cmdenv-log-finalization] = \textit{<bool>}\\
    \textit{Per-simulation-run setting.}\\
    Specifies whether to enable logging during network finalization. The
    default depends on whether the simulation is run in express mode or not
    (see \ttt{cmdenv-{\allowbreak}express-{\allowbreak}mode}).
\item[cmdenv-log-initialization] = \textit{<bool>}\\
    \textit{Per-simulation-run setting.}\\
    Specifies whether to enable logging during network initialization. The
    default depends on whether the simulation is run in express mode or not
    (see \ttt{cmdenv-{\allowbreak}express-{\allowbreak}mode}).
\item[**.cmdenv-log-level] = \textit{<string>}, default: \ttt{TRACE}\\
    \textit{Per-object setting for modules.}\\
    Specifies the per-component level of detail recorded by log statements,
    output below the specified level is omitted. Available values are (case
    insensitive): \ttt{off}, \ttt{fatal}, \ttt{error}, \ttt{warn}, \ttt{info},
    \ttt{detail}, \ttt{debug} or \ttt{trace}. Note that the level of detail is
    also controlled by the globally specified runtime log level and the
    \ttt{COMPILETIME\_LOGLEVEL} macro that is used to completely remove log
    statements from the executable.
\item[cmdenv-log-prefix] = \textit{<string>}, default: \ttt{[\%l]	}\\
    \textit{Per-simulation-run setting.}\\
    Specifies the format string that determines the prefix of each log line.
    The format string may contain format directives in the syntax \ttt{\%x} (a
    \ttt{\%} followed by a single format character).  For example \ttt{\%l}
    stands for log level, and \ttt{\%J} for source component. See the manual
    for the list of available format characters.
\item[cmdenv-log-simulation] = \textit{<bool>}\\
    \textit{Per-simulation-run setting.}\\
    Specifies whether to enable logging during the simulation, i.e. event
    processing. The default depends on whether the simulation is run in express
    mode or not (see \ttt{cmdenv-{\allowbreak}express-{\allowbreak}mode}).
\item[cmdenv-output-file] = \textit{<filename>}, default: \ttt{\$\{{\allowbreak}resultdir\}{\allowbreak}/{\allowbreak}\$\{{\allowbreak}configname\}{\allowbreak}-{\allowbreak}\$\{{\allowbreak}iterationvarsf\}{\allowbreak}\#\$\{{\allowbreak}repetition\}{\allowbreak}.{\allowbreak}out}\\
    \textit{Per-simulation-run setting.}\\
    When
    \ttt{cmdenv-{\allowbreak}record-{\allowbreak}output={\allowbreak}true}:
    file name to redirect standard output to. See also
    \ttt{fname-{\allowbreak}append-{\allowbreak}host}.
\item[cmdenv-performance-display] = \textit{<bool>}, default: \ttt{true}\\
    \textit{Per-simulation-run setting.}\\
    When \ttt{cmdenv-{\allowbreak}express-{\allowbreak}mode={\allowbreak}true}:
    print detailed performance information. Turning it on results in a 3-line
    entry printed on each update, containing ev/sec, simsec/sec, ev/simsec,
    number of messages created/still present/currently scheduled in FES.
\item[cmdenv-progress-updates] = \textit{<bool>}, default: \ttt{true}\\
    \textit{Per-simulation-run setting.}\\
    When \ttt{cmdenv-{\allowbreak}express-{\allowbreak}mode={\allowbreak}true}:
    print a periodic status update on the progress of the simulation.
\item[cmdenv-redirect-output] = \textit{<bool>}, default: \ttt{false}\\
    \textit{Per-simulation-run setting.}\\
    Causes Cmdenv to redirect standard output of simulation runs to a file or
    separate files per run. This option can be useful with running simulation
    campaigns (e.g. using opp\_runall), and also with parallel simulation. See
    also: \ttt{cmdenv-{\allowbreak}output-{\allowbreak}file},
    \ttt{fname-{\allowbreak}append-{\allowbreak}host}.
\item[cmdenv-runs-to-execute] = \textit{<string>}\\
    \textit{Global setting (applies to all simulation runs).}\\
    Specifies which runs to execute from the selected configuration (see
    \ttt{cmdenv-{\allowbreak}config-{\allowbreak}name} option). It accepts a
    filter expression of iteration variables such as \ttt{\$num\-Hosts>10
    {\allowbreak}\&\& {\allowbreak}\$iatime={\allowbreak}={\allowbreak}1s}, or
    a comma-separated list of run numbers or run number ranges, e.g.
    \ttt{1,{\allowbreak}3..{\allowbreak}4,{\allowbreak}7..{\allowbreak}9}. If
    the value is missing, Cmdenv executes all runs in the selected
    configuration. The \ttt{-{\allowbreak}r} command line option overrides this
    setting.
\item[cmdenv-status-frequency] = \textit{<double>}, unit=\ttt{s}, default: \ttt{2s}\\
    \textit{Per-simulation-run setting.}\\
    When \ttt{cmdenv-{\allowbreak}express-{\allowbreak}mode={\allowbreak}true}:
    print status update every n seconds.
\item[cmdenv-stop-batch-on-error] = \textit{<bool>}, default: \ttt{true}\\
    \textit{Per-simulation-run setting.}\\
    Decides whether Cmdenv should skip the rest of the runs when an error
    occurs during the execution of one run.
\item[config-recording] = \textit{<custom>}, default: \ttt{all}\\
    \textit{Per-simulation-run setting.}\\
    Selects the set of config options to save into result files. This option
    can help reduce the size of result files, which is especially useful in the
    case of large simulation campaigns. Possible values: all, none, config,
    params, essentials, globalconfig
\item[configuration-class] = \textit{<string>}\\
    \textit{Global setting (applies to all simulation runs).}\\
    Part of the Envir plugin mechanism: selects the class from which all
    configuration information will be obtained. This option lets you replace
    omnetpp.ini with some other implementation, e.g. database input. The
    simulation program still has to bootstrap from an omnetpp.ini (which
    contains the configuration-class setting). The class should implement the
    \ttt{cConfiguration\-Ex} interface.
\item[constraint] = \textit{<string>}\\
    \textit{Per-simulation-run setting.}\\
    For scenarios. Contains an expression that iteration variables
    (\ttt{\$\{{\allowbreak}\}{\allowbreak}} syntax) must satisfy for that
    simulation to run. Example: \ttt{\$i {\allowbreak}< {\allowbreak}\$j+1}.
\item[cpu-time-limit] = \textit{<double>}, unit=\ttt{s}\\
    \textit{Per-simulation-run setting.}\\
    Stops the simulation when CPU usage has reached the given limit. The
    default is no limit. Note: To reduce per-event overhead, this time limit is
    only checked every N events (by default, N=1024).
\item[debug-on-errors] = \textit{<bool>}, default: \ttt{false}\\
    \textit{Per-simulation-run setting.}\\
    When set to true, runtime errors will cause the simulation program to break
    into the C++ debugger (if the simulation is running under one, or
    just-in-time debugging is activated). Once in the debugger, you can view
    the stack trace or examine variables.
\item[debug-statistics-recording] = \textit{<bool>}, default: \ttt{false}\\
    \textit{Per-simulation-run setting.}\\
    Turns on the printing of debugging information related to statistics
    recording (\ttt{@{\allowbreak}statistic} properties)
\item[debugger-attach-command] = \textit{<string>}\\
    \textit{Global setting (applies to all simulation runs).}\\
    The command line to launch the debugger. It must contain exactly one
    percent sign, as \ttt{\%u}, which will be replaced by the PID of this
    process. The command must not block (i.e. it should end in \ttt{\&} on
    Unix-like systems). It will be executed by the default system shell (on
    Windows, usually cmd.exe). Default on this platform: \ttt{opp\_ide
    {\allowbreak}omnetpp:{\allowbreak}/{\allowbreak}/{\allowbreak}dsp/{\allowbreak}debugger/{\allowbreak}attach?pid={\allowbreak}\%u}.
    This default can be overridden with the \ttt{OMNETPP\_DEBUGGER\_COMMAND}
    environment variable.
\item[debugger-attach-on-error] = \textit{<bool>}, default: \ttt{false}\\
    \textit{Global setting (applies to all simulation runs).}\\
    When set to true, runtime errors and crashes will trigger an external
    debugger to be launched (if not already present), allowing you to perform
    just-in-time debugging on the simulation process. The debugger command is
    configurable. Note that debugging (i.e. attaching to) a non-child process
    needs to be explicitly enabled on some systems, e.g. Ubuntu.
\item[debugger-attach-on-startup] = \textit{<bool>}, default: \ttt{false}\\
    \textit{Global setting (applies to all simulation runs).}\\
    When set to true, the simulation program will launch an external debugger
    attached to it (if not already present), allowing you to set breakpoints
    before proceeding. The debugger command is configurable. Note that
    debugging (i.e. attaching to) a non-child process needs to be explicitly
    enabled on some systems, e.g. Ubuntu.
\item[debugger-attach-wait-time] = \textit{<double>}, unit=\ttt{s}, default: \ttt{20s}\\
    \textit{Global setting (applies to all simulation runs).}\\
    An interval to wait after launching the external debugger, to give the
    debugger time to start up and attach to the simulation process.
\item[description] = \textit{<string>}\\
    \textit{Per-simulation-run setting.}\\
    Descriptive name for the given simulation configuration. Descriptions get
    displayed in the run selection dialog.
\item[**.display-name] = \textit{<string>}\\
    \textit{Per-object setting for modules.}\\
    Specifies a display name for the module, which is shown e.g. in Qtenv's
    graphical module view.
\item[**.display-string] = \textit{<string>}\\
    \textit{Per-object setting for modules and channels.}\\
    Additional display string for the module/channel; it will be merged into
    the display string given via \ttt{@{\allowbreak}display} properties, and
    override its content.
\item[eventlog-file] = \textit{<filename>}, default: \ttt{\$\{{\allowbreak}resultdir\}{\allowbreak}/{\allowbreak}\$\{{\allowbreak}configname\}{\allowbreak}-{\allowbreak}\$\{{\allowbreak}iterationvarsf\}{\allowbreak}\#\$\{{\allowbreak}repetition\}{\allowbreak}.{\allowbreak}elog}\\
    \textit{Per-simulation-run setting.}\\
    Name of the eventlog file to generate.
\item[eventlog-index-frequency] = \textit{<double>}, unit=\ttt{B}, default: \ttt{1 {\allowbreak}Mi\-B}\\
    \textit{Per-simulation-run setting.}\\
    The eventlog file contains incremental snapshots called index. An index is
    much smaller than a full snapshot, but it only contains the differences
    since the last index.
\item[eventlog-max-size] = \textit{<double>}, unit=\ttt{B}, default: \ttt{10 {\allowbreak}Gi\-B}\\
    \textit{Per-simulation-run setting.}\\
    Specify the maximum size of the eventlog file in bytes. The eventlog file
    is automatically truncated when this limit is reached.
\item[eventlog-message-detail-pattern] = \textit{<custom>}\\
    \textit{Per-simulation-run setting.}\\
    A list of patterns separated by '|' character which will be used to write
    message detail information into the eventlog for each message sent during
    the simulation. The message detail will be presented in the sequence chart
    tool. Each pattern starts with an object pattern optionally followed by ':'
    character and a comma separated list of field patterns. In both patterns
    and/or/not/* and various field match expressions can be used. The object
    pattern matches to class name, the field pattern matches to field name by
    default.\\
     
    \ttt{EVENTLOG-{\allowbreak}MESSAGE-{\allowbreak}DETAIL-{\allowbreak}PATTERN
    {\allowbreak}:{\allowbreak}={\allowbreak} {\allowbreak}(
    {\allowbreak}DETAIL-{\allowbreak}PATTERN {\allowbreak}'|' {\allowbreak})*
    {\allowbreak}DETAIL\_PATTERN}\\
      \ttt{DETAIL-{\allowbreak}PATTERN
    {\allowbreak}:{\allowbreak}={\allowbreak}
    {\allowbreak}OBJECT-{\allowbreak}PATTERN {\allowbreak}[
    {\allowbreak}':{\allowbreak}' {\allowbreak}FIELD-{\allowbreak}PATTERNS
    {\allowbreak}]}\\
      \ttt{OBJECT-{\allowbreak}PATTERN
    {\allowbreak}:{\allowbreak}={\allowbreak}
    {\allowbreak}MATCH-{\allowbreak}EXPRESSION}\\
      \ttt{FIELD-{\allowbreak}PATTERNS
    {\allowbreak}:{\allowbreak}={\allowbreak} {\allowbreak}(
    {\allowbreak}FIELD-{\allowbreak}PATTERN {\allowbreak}',{\allowbreak}'
    {\allowbreak})* {\allowbreak}FIELD\_PATTERN}\\
      \ttt{FIELD-{\allowbreak}PATTERN {\allowbreak}:{\allowbreak}={\allowbreak}
    {\allowbreak}MATCH-{\allowbreak}EXPRESSION}\\
    Examples:\\
      \ttt{*}: captures all fields of all messages\\
      \ttt{*Frame:{\allowbreak}*Address,{\allowbreak}*Id}: captures all fields
    named somethingAddress and somethingId from messages of any class named
    somethingFrame\\
     
    \ttt{My\-Message:{\allowbreak}declared\-On={\allowbreak}{\textasciitilde}My\-Message}:
    captures instances of MyMessage recording the fields declared on the
    MyMessage class\\
      \ttt{*:{\allowbreak}(not
    {\allowbreak}declared\-On={\allowbreak}{\textasciitilde}cMessage
    {\allowbreak}and {\allowbreak}not
    {\allowbreak}declared\-On={\allowbreak}{\textasciitilde}cNamed\-Object
    {\allowbreak}and {\allowbreak}not
    {\allowbreak}declared\-On={\allowbreak}{\textasciitilde}cObject)}: records
    user-defined fields from all messages
\item[eventlog-min-truncated-size] = \textit{<double>}, unit=\ttt{B}, default: \ttt{1 {\allowbreak}Gi\-B}\\
    \textit{Per-simulation-run setting.}\\
    Specify the minimum size of the eventlog file in bytes after the file is
    truncated. Truncation means older events are discarded while newer ones are
    kept.
\item[eventlog-options] = \textit{<custom>}\\
    \textit{Per-simulation-run setting.}\\
    Allows for reducing the size of the eventlog file by recording only
    specific types of content. Specify a comma-separated subset of the
    following keywords: text, message, module, methodcall, displaystring and
    custom. By default, all categories are enabled.
\item[eventlog-recording-intervals] = \textit{<custom>}\\
    \textit{Per-simulation-run setting.}\\
    Simulation time interval(s) when events should be recorded. Syntax:
    \ttt{[<from>]..{\allowbreak}[<to>],{\allowbreak}...{\allowbreak}} That is,
    both start and end of an interval are optional, and intervals are separated
    by comma. Example: \ttt{..{\allowbreak}10.{\allowbreak}2,{\allowbreak}
    {\allowbreak}22.{\allowbreak}2..{\allowbreak}100,{\allowbreak}
    {\allowbreak}233.{\allowbreak}3..{\allowbreak}}
\item[eventlog-snapshot-frequency] = \textit{<double>}, unit=\ttt{B}, default: \ttt{100 {\allowbreak}Mi\-B}\\
    \textit{Per-simulation-run setting.}\\
    The eventlog file contains periodically recorded snapshots. Each one
    describes the complete simulation state at a specific event. Snapshots help
    various tools to handle large eventlog files more efficiently. Specifying
    greater value means less help, while smaller value means bigger eventlog
    files.
\item[eventlogmanager-class] = \textit{<string>}, default: \ttt{omnetpp::{\allowbreak}envir::{\allowbreak}Eventlog\-File\-Manager}\\
    \textit{Per-simulation-run setting.}\\
    Part of the Envir plugin mechanism: selects the eventlog manager class to
    be used to record data. The class has to implement the
    \ttt{cIEventlog\-Manager} interface.
\item[experiment-label] = \textit{<string>}, default: \ttt{\$\{{\allowbreak}configname\}{\allowbreak}}\\
    \textit{Per-simulation-run setting.}\\
    Identifies the simulation experiment (which consists of several,
    potentially repeated measurements). This string gets recorded into result
    files, and may be referred to during result analysis.
\item[extends] = \textit{<string>}\\
    \textit{Per-simulation-run setting.}\\
    Name of the configuration this section is based on. Entries from that
    section will be inherited and can be overridden. In other words,
    configuration lookups will fall back to the base section.
\item[fingerprint] = \textit{<string>}\\
    \textit{Per-simulation-run setting.}\\
    The expected fingerprints of the simulation. If you need multiple
    fingerprints, separate them with commas. When provided, the fingerprints
    will be calculated from the specified properties of simulation events,
    messages, and statistics during execution, and checked against the provided
    values. Fingerprints are suitable for crude regression tests. As
    fingerprints occasionally differ across platforms, more than one value can
    be specified for a single fingerprint, separated by spaces, and a match
    with any of them will be accepted. To obtain a fingerprint, enter a dummy
    value (such as \ttt{0000}), and run the simulation.
\item[fingerprint-events] = \textit{<string>}, default: \ttt{*}\\
    \textit{Per-simulation-run setting.}\\
    Configures the fingerprint calculator to consider only certain events. The
    value is a pattern that will be matched against the event name by default.
    It may also be an expression containing pattern matching characters, field
    access, and logical operators. The default setting is '*' which includes
    all events in the calculated fingerprint. If you configured multiple
    fingerprints, separate filters with commas.
\item[fingerprint-ingredients] = \textit{<string>}, default: \ttt{tplx}\\
    \textit{Per-simulation-run setting.}\\
    Specifies the list of ingredients to be taken into account for fingerprint
    computation. Each character corresponds to one ingredient: 'e' event
    number, 't' simulation time, 'n' message (event) full name, 'c' message
    (event) class name, 'k' message kind, 'l' message bit length, 'o' message
    control info class name, 'd' message data, 'i' module id, 'm' module full
    name, 'p' module full path, 'a' module class name, 'r' random numbers
    drawn, 's' scalar results, 'z' statistic results, 'v' vector results, 'x'
    extra data provided by modules. Note: ingredients specified in an expected
    fingerprint (characters after the '/' in the fingerprint value) take
    precedence over this setting. If you configured multiple fingerprints,
    separate ingredients with commas.
\item[fingerprint-modules] = \textit{<string>}, default: \ttt{*}\\
    \textit{Per-simulation-run setting.}\\
    Configures the fingerprint calculator to consider only certain modules. The
    value is a pattern that will be matched against the module full path by
    default. It may also be an expression containing pattern matching
    characters, field access, and logical operators. The default setting is '*'
    which includes all events in all modules in the calculated fingerprint. If
    you configured multiple fingerprints, separate filters with commas.
\item[fingerprint-results] = \textit{<string>}, default: \ttt{*}\\
    \textit{Per-simulation-run setting.}\\
    Configures the fingerprint calculator to consider only certain results. The
    value is a pattern that will be matched against the result full path by
    default. It may also be an expression containing pattern matching
    characters, field access, and logical operators. The default setting is '*'
    which includes all results in all modules in the calculated fingerprint. If
    you configured multiple fingerprints, separate filters with commas.
\item[fingerprintcalculator-class] = \textit{<string>}, default: \ttt{omnetpp::{\allowbreak}cSingle\-Fingerprint\-Calculator}\\
    \textit{Per-simulation-run setting.}\\
    Part of the Envir plugin mechanism: selects the fingerprint calculator
    class to be used to calculate the simulation fingerprint. The class has to
    implement the \ttt{cFingerprint\-Calculator} interface.
\item[fname-append-host] = \textit{<bool>}\\
    \textit{Global setting (applies to all simulation runs).}\\
    Turning it on will cause the host name and process Id to be appended to the
    names of output files (e.g. omnetpp.vec, omnetpp.sca). This is especially
    useful with distributed simulation. The default value is true if parallel
    simulation is enabled, false otherwise.
\item[futureeventset-class] = \textit{<string>}, default: \ttt{omnetpp::{\allowbreak}cEvent\-Heap}\\
    \textit{Per-simulation-run setting.}\\
    Part of the Envir plugin mechanism: selects the class for storing the
    future events in the simulation. The class has to implement the
    \ttt{cFuture\-Event\-Set} interface.
\item[image-path] = \textit{<path>}, default: \ttt{.{\allowbreak}/{\allowbreak}images}\\
    \textit{Global setting (applies to all simulation runs).}\\
    A semicolon-separated list of directories that contain module icons and
    other resources. This list will be concatenated with the contents of the
    \ttt{OMNETPP\_IMAGE\_PATH} environment variable or with a compile-time,
    hardcoded image path if the environment variable is empty.
\item[iteration-nesting-order] = \textit{<string>}\\
    \textit{Per-simulation-run setting.}\\
    Specifies the loop nesting order for iteration variables
    (\ttt{\$\{{\allowbreak}\}{\allowbreak}} syntax). The value is a
    comma-separated list of iteration variables; the list may also contain at
    most one asterisk. Variables that are not explicitly listed will be
    inserted at the position of the asterisk, or appended to the list if there
    is no asterisk. The first variable will become the outermost loop, and the
    last one the innermost loop. Example:
    \ttt{repetition,{\allowbreak}num\-Hosts,{\allowbreak}*,{\allowbreak}ia\-Time}.
\item[load-libs] = \textit{<filenames>}\\
    \textit{Global setting (applies to all simulation runs).}\\
    A space-separated list of dynamic libraries to be loaded on startup. The
    libraries should be given without the \ttt{.{\allowbreak}dll} or
    \ttt{.{\allowbreak}so} suffix -- that will be automatically appended.
\item[max-module-nesting] = \textit{<int>}, default: \ttt{50}\\
    \textit{Per-simulation-run setting.}\\
    The maximum allowed depth of submodule nesting. This is used to catch
    accidental infinite recursions in NED.
\item[measurement-label] = \textit{<string>}, default: \ttt{\$\{{\allowbreak}iterationvars\}{\allowbreak}}\\
    \textit{Per-simulation-run setting.}\\
    Identifies the measurement within the experiment. This string gets recorded
    into result files, and may be referred to during result analysis.
\item[**.module-eventlog-recording] = \textit{<bool>}, default: \ttt{true}\\
    \textit{Per-object setting for simple modules.}\\
    Enables recording events on a per module basis. This is meaningful for
    simple modules only. Usage:
    \ttt{<module-{\allowbreak}full-{\allowbreak}path>.{\allowbreak}module-{\allowbreak}eventlog-{\allowbreak}recording={\allowbreak}true/{\allowbreak}false}.
    Examples:
    \ttt{**.{\allowbreak}router[10..{\allowbreak}20].{\allowbreak}**.{\allowbreak}module-{\allowbreak}eventlog-{\allowbreak}recording
    {\allowbreak}={\allowbreak} {\allowbreak}true};
    \ttt{**.{\allowbreak}module-{\allowbreak}eventlog-{\allowbreak}recording
    {\allowbreak}={\allowbreak} {\allowbreak}false}
\item[ned-package-exclusions] = \textit{<custom>}\\
    \textit{Global setting (applies to all simulation runs).}\\
    A semicolon-separated list of NED packages to be excluded when loading NED
    files. Sub-packages of excluded ones are also excluded. Additional items
    may be specified via the \ttt{-{\allowbreak}x} command-line option and the
    \ttt{OMNETPP\_NED\_PACKAGE\_EXCLUSIONS} environment variable.
\item[ned-path] = \textit{<path>}\\
    \textit{Global setting (applies to all simulation runs).}\\
    A semicolon-separated list of directories. The directories will be regarded
    as roots of the NED package hierarchy, and all NED files will be loaded
    from their subdirectory trees. This option is normally left empty, as the
    OMNeT++ IDE sets the NED path automatically, and for simulations started
    outside the IDE it is more convenient to specify it via command-line option
    (-n) or via environment variable (OMNETPP\_NED\_PATH, NEDPATH).
\item[network] = \textit{<string>}\\
    \textit{Per-simulation-run setting.}\\
    The name of the network to be simulated.  The package name can be omitted
    if the ini file is in the same directory as the NED file that contains the
    network.
\item[num-rngs] = \textit{<int>}, default: \ttt{1}\\
    \textit{Per-simulation-run setting.}\\
    The number of random number generators.
\item[output-scalar-db-commit-freq] = \textit{<int>}, default: \ttt{100000}\\
    \textit{Global setting (applies to all simulation runs).}\\
    Used with SqliteOutputScalarManager: COMMIT every n INSERTs.
\item[output-scalar-file] = \textit{<filename>}, default: \ttt{\$\{{\allowbreak}resultdir\}{\allowbreak}/{\allowbreak}\$\{{\allowbreak}configname\}{\allowbreak}-{\allowbreak}\$\{{\allowbreak}iterationvarsf\}{\allowbreak}\#\$\{{\allowbreak}repetition\}{\allowbreak}.{\allowbreak}sca}\\
    \textit{Per-simulation-run setting.}\\
    Name for the output scalar file.
\item[output-scalar-file-append] = \textit{<bool>}, default: \ttt{false}\\
    \textit{Per-simulation-run setting.}\\
    What to do when the output scalar file already exists: append to it
    (OMNeT++ 3.x behavior), or delete it and begin a new file (default).
\item[output-scalar-precision] = \textit{<int>}, default: \ttt{14}\\
    \textit{Per-simulation-run setting.}\\
    The number of significant digits for recording data into the output scalar
    file. The maximum value is {\textasciitilde}15 (IEEE double precision).
    This has no effect on SQLite recording, as it stores values as 8-byte IEEE
    floating point numbers.
\item[output-vector-db-indexing] = \textit{<custom>}, default: \ttt{skip}\\
    \textit{Global setting (applies to all simulation runs).}\\
    Whether and when to add an index to the 'vectordata' table in SQLite output
    vector files. Possible values: skip, ahead, after
\item[output-vector-file] = \textit{<filename>}, default: \ttt{\$\{{\allowbreak}resultdir\}{\allowbreak}/{\allowbreak}\$\{{\allowbreak}configname\}{\allowbreak}-{\allowbreak}\$\{{\allowbreak}iterationvarsf\}{\allowbreak}\#\$\{{\allowbreak}repetition\}{\allowbreak}.{\allowbreak}vec}\\
    \textit{Per-simulation-run setting.}\\
    Name for the output vector file.
\item[output-vector-file-append] = \textit{<bool>}, default: \ttt{false}\\
    \textit{Per-simulation-run setting.}\\
    What to do when the output vector file already exists: append to it, or
    delete it and begin a new file (default). Note:
    \ttt{cIndexed\-File\-Output\-Vector\-Manager} currently does not support
    appending.
\item[output-vector-precision] = \textit{<int>}, default: \ttt{14}\\
    \textit{Per-simulation-run setting.}\\
    The number of significant digits for recording data into the output vector
    file. The maximum value is {\textasciitilde}15 (IEEE double precision).
    This setting has no effect on SQLite recording (it stores values as 8-byte
    IEEE floating point numbers), and for the "time" column which is
    represented as fixed-point numbers and always get recorded precisely.
\item[output-vectors-memory-limit] = \textit{<double>}, unit=\ttt{B}, default: \ttt{16Mi\-B}\\
    \textit{Per-simulation-run setting.}\\
    Total memory that can be used for buffering output vectors. Larger values
    produce less fragmented vector files (i.e. cause vector data to be grouped
    into larger chunks), and therefore allow more efficient processing later.
    There is also a per-vector limit, see
    \ttt{**.{\allowbreak}vector-{\allowbreak}buffer}.
\item[outputscalarmanager-class] = \textit{<string>}, default: \ttt{omnetpp::{\allowbreak}envir::{\allowbreak}Omnetpp\-Output\-Scalar\-Manager}\\
    \textit{Per-simulation-run setting.}\\
    Part of the Envir plugin mechanism: selects the output scalar manager class
    to be used to record data passed to recordScalar(). The class has to
    implement the \ttt{cIOutput\-Scalar\-Manager} interface.
\item[outputvectormanager-class] = \textit{<string>}, default: \ttt{omnetpp::{\allowbreak}envir::{\allowbreak}Omnetpp\-Output\-Vector\-Manager}\\
    \textit{Per-simulation-run setting.}\\
    Part of the Envir plugin mechanism: selects the output vector manager class
    to be used to record data from output vectors. The class has to implement
    the \ttt{cIOutput\-Vector\-Manager} interface.
\item[parallel-simulation] = \textit{<bool>}, default: \ttt{false}\\
    \textit{Global setting (applies to all simulation runs).}\\
    Enables parallel distributed simulation.
\item[**.param-record-as-scalar] = \textit{<bool>}, default: \ttt{false}\\
    \textit{Per-object setting for module/channel parameters.}\\
    Applicable to module parameters: specifies whether the module parameter
    should be recorded into the output scalar file. Set it for parameters whose
    value you will need for result analysis.
\item[**.param-recording] = \textit{<bool>}, default: \ttt{true}\\
    \textit{Per-object setting for module/channel parameters.}\\
    Whether the matching module (and channel) parameters should be recorded.\\
    Usage:
    \ttt{<module-{\allowbreak}full-{\allowbreak}path>.{\allowbreak}<parameter-{\allowbreak}name>.{\allowbreak}param-{\allowbreak}recording={\allowbreak}true/{\allowbreak}false}.\\
    Example:
    \ttt{**.{\allowbreak}app.{\allowbreak}pk\-Len.{\allowbreak}param-{\allowbreak}recording={\allowbreak}true}
\item[parameter-mutability-check] = \textit{<bool>}, default: \ttt{true}\\
    \textit{Per-simulation-run setting.}\\
    Setting to false will disable errors raised when trying to change the
    values of module/channel parameters not marked as @mutable. This is
    primarily a compatibility setting intended to facilitate running simulation
    models that were not yet annotated with @mutable.
\item[parsim-communications-class] = \textit{<string>}, default: \ttt{omnetpp::{\allowbreak}cFile\-Communications}\\
    \textit{Global setting (applies to all simulation runs).}\\
    If \ttt{parallel-{\allowbreak}simulation={\allowbreak}true}, it selects the
    class that implements communication between partitions. The class must
    implement the \ttt{cParsim\-Communications} interface.
\item[parsim-debug] = \textit{<bool>}, default: \ttt{true}\\
    \textit{Global setting (applies to all simulation runs).}\\
    With \ttt{parallel-{\allowbreak}simulation={\allowbreak}true}: turns on
    printing of log messages from the parallel simulation code.
\item[parsim-filecommunications-prefix] = \textit{<string>}, default: \ttt{comm/{\allowbreak}}\\
    \textit{Global setting (applies to all simulation runs).}\\
    When \ttt{cFile\-Communications} is selected as parsim communications
    class: specifies the prefix (directory+potential filename prefix) for
    creating the files for cross-partition messages.
\item[parsim-filecommunications-preserve-read] = \textit{<bool>}, default: \ttt{false}\\
    \textit{Global setting (applies to all simulation runs).}\\
    When \ttt{cFile\-Communications} is selected as parsim communications
    class: specifies that consumed files should be moved into another directory
    instead of being deleted.
\item[parsim-filecommunications-read-prefix] = \textit{<string>}, default: \ttt{comm/{\allowbreak}read/{\allowbreak}}\\
    \textit{Global setting (applies to all simulation runs).}\\
    When \ttt{cFile\-Communications} is selected as parsim communications
    class: specifies the prefix (directory) where files will be moved after
    having been consumed.
\item[parsim-idealsimulationprotocol-tablesize] = \textit{<int>}, default: \ttt{100000}\\
    \textit{Global setting (applies to all simulation runs).}\\
    When \ttt{cIdeal\-Simulation\-Protocol} is selected as parsim
    synchronization class: specifies the memory buffer size for reading the ISP
    event trace file.
\item[parsim-mpicommunications-mpibuffer] = \textit{<int>}\\
    \textit{Global setting (applies to all simulation runs).}\\
    When \ttt{cMPICommunications} is selected as parsim communications class:
    specifies the size of the MPI communications buffer. The default is to
    calculate a buffer size based on the number of partitions.
\item[parsim-namedpipecommunications-prefix] = \textit{<string>}, default: \ttt{comm/{\allowbreak}}\\
    \textit{Global setting (applies to all simulation runs).}\\
    When \ttt{cNamed\-Pipe\-Communications} is selected as parsim
    communications class: selects the prefix (directory+potential filename
    prefix) where name pipes are created in the file system.
\item[parsim-nullmessageprotocol-laziness] = \textit{<double>}, default: \ttt{0.{\allowbreak}5}\\
    \textit{Global setting (applies to all simulation runs).}\\
    When \ttt{cNull\-Message\-Protocol} is selected as parsim synchronization
    class: specifies the laziness of sending null messages. Values in the range
    \ttt{[0,{\allowbreak}1)} are accepted. Laziness=0 causes null messages to
    be sent out immediately as a new EOT is learned, which may result in
    excessive null message traffic.
\item[parsim-nullmessageprotocol-lookahead-class] = \textit{<string>}, default: \ttt{cLink\-Delay\-Lookahead}\\
    \textit{Global setting (applies to all simulation runs).}\\
    When \ttt{cNull\-Message\-Protocol} is selected as parsim synchronization
    class: specifies the C++ class that calculates lookahead. The class should
    subclass from \ttt{cNMPLookahead}.
\item[parsim-num-partitions] = \textit{<int>}\\
    \textit{Global setting (applies to all simulation runs).}\\
    If \ttt{parallel-{\allowbreak}simulation={\allowbreak}true}, it tells the
    number of parallel processes to use. This value must be in agreement with
    the number of simulator instances launched, e.g. with the
    \ttt{-{\allowbreak}n} or \ttt{-{\allowbreak}np} command-line option
    specified to the \ttt{mpirun} program.
\item[parsim-synchronization-class] = \textit{<string>}, default: \ttt{omnetpp::{\allowbreak}cNull\-Message\-Protocol}\\
    \textit{Global setting (applies to all simulation runs).}\\
    If \ttt{parallel-{\allowbreak}simulation={\allowbreak}true}, it selects the
    parallel simulation algorithm. The class must implement the
    \ttt{cParsim\-Synchronizer} interface.
\item[**.partition-id] = \textit{<string>}\\
    \textit{Per-object setting for modules.}\\
    With parallel simulation: in which partition the module should be
    instantiated. Specify numeric partition ID, or a comma-separated list of
    partition IDs for compound modules that span across multiple partitions.
    Ranges (\ttt{5..{\allowbreak}9}) and \ttt{*} (=all) are accepted too.
\item[print-instantiated-ned-types] = \textit{<bool>}, default: \ttt{false}\\
    \textit{Per-simulation-run setting.}\\
    When set to true, the simulation will print the names of all instantiated
    NED types at the end of the simulation. This is useful for coverage tests.
\item[print-undisposed] = \textit{<bool>}, default: \ttt{true}\\
    \textit{Per-simulation-run setting.}\\
    Whether to report objects left (that is, not deallocated by simple module
    destructors) after network cleanup.
\item[print-unused-parameters] = \textit{<bool>}, default: \ttt{false}\\
    \textit{Per-simulation-run setting.}\\
    When set to true, the simulation will print the names of all parameters
    that have not been accessed during simulation, i.e. did not have any
    effect, pointing to possible deficiencies in the model.
\item[qtenv-default-config] = \textit{<string>}\\
    \textit{Global setting (applies to all simulation runs).}\\
    Specifies which config Qtenv should set up automatically on startup. The
    default is to ask the user.
\item[qtenv-default-run] = \textit{<string>}\\
    \textit{Global setting (applies to all simulation runs).}\\
    Specifies which run (of the default config, see
    \ttt{qtenv-{\allowbreak}default-{\allowbreak}config}) Qtenv should set up
    automatically on startup. A run filter is also accepted. The default is to
    ask the user.
\item[qtenv-extra-stack] = \textit{<double>}, unit=\ttt{B}, default: \ttt{80Ki\-B}\\
    \textit{Global setting (applies to all simulation runs).}\\
    Specifies the extra amount of stack that is reserved for each
    \ttt{activity()} simple module when the simulation is run under Qtenv.
\item[qtenv-identicon-seed] = \textit{<string>}\\
    \textit{Per-simulation-run setting.}\\
    Overrides the default (computed) identicon seed for the current run. The
    identicon is a small image that is shown in the upper right corner of the
    Qtenv main window, and its purpose is help the user visually identify the
    window if there are multiple Qtenv simulations running at the same time.
    The image is generated from a seed. By default, the seed is computed from
    various properties of the simulation (config name, run number,
    configuration contents, etc) that ensures that different simulations result
    in different images, but multiple launches of the same simulation result in
    the same identicon. This configuration option allows the user to override
    this default behavior and force a specific seed.
\item[real-time-limit] = \textit{<double>}, unit=\ttt{s}\\
    \textit{Per-simulation-run setting.}\\
    Stops the simulation after the specified amount of time has elapsed. The
    default is no limit. Note: To reduce per-event overhead, this time limit is
    only checked every N events (by default, N=1024).
\item[realtimescheduler-scaling] = \textit{<double>}\\
    \textit{Global setting (applies to all simulation runs).}\\
    When cRealTimeScheduler is selected as scheduler class: ratio of simulation
    time to real time. For example,
    \ttt{realtimescheduler-{\allowbreak}scaling={\allowbreak}2} will cause
    simulation time to progress twice as fast as runtime.
\item[record-eventlog] = \textit{<bool>}, default: \ttt{false}\\
    \textit{Per-simulation-run setting.}\\
    Enables recording an eventlog file, which can be later visualized on a
    sequence chart. See \ttt{eventlog-{\allowbreak}file} option too.
\item[record-scalar-results] = \textit{<bool>}, default: \ttt{true}\\
    \textit{Per-simulation-run setting.}\\
    Enables writing an output scalar (\ttt{.{\allowbreak}sca}) file. See the
    \ttt{output-{\allowbreak}scalar-{\allowbreak}file} option too.
\item[record-vector-results] = \textit{<bool>}, default: \ttt{true}\\
    \textit{Per-simulation-run setting.}\\
    Enables writing an output vector (\ttt{.{\allowbreak}vec}) file. See the
    \ttt{output-{\allowbreak}vector-{\allowbreak}file} option too.
\item[repeat] = \textit{<int>}, default: \ttt{1}\\
    \textit{Per-simulation-run setting.}\\
    For scenarios. Specifies how many replications should be done with the same
    parameters (iteration variables). This is typically used to perform
    multiple runs with different random number seeds. The loop variable is
    available as \ttt{\$\{{\allowbreak}repetition\}{\allowbreak}}. See also:
    \ttt{seed-{\allowbreak}set} key.
\item[replication-label] = \textit{<string>}, default: \ttt{\#\$\{{\allowbreak}repetition\}{\allowbreak}}\\
    \textit{Per-simulation-run setting.}\\
    Identifies one replication of a measurement (see \ttt{repeat} and
    \ttt{measurement-{\allowbreak}label} options as well). This string gets
    recorded into result files, and may be referred to during result analysis.
\item[result-dir] = \textit{<string>}, default: \ttt{results}\\
    \textit{Per-simulation-run setting.}\\
    Base value for the \ttt{\$\{{\allowbreak}resultdir\}{\allowbreak}}
    variable, which is used as the default directory for result files (output
    vector file, output scalar file, eventlog file, etc.). See also the
    \ttt{resultdir-{\allowbreak}subdivision} config option.
\item[**.result-recording-modes] = \textit{<string>}, default: \ttt{default}\\
    \textit{Per-object setting for statistics (@statistic).}\\
    Defines how to calculate results from the matching
    \ttt{@{\allowbreak}statistic}.\\
    Usage:
    \ttt{<module-{\allowbreak}full-{\allowbreak}path>.{\allowbreak}<statistic-{\allowbreak}name>.{\allowbreak}result-{\allowbreak}recording-{\allowbreak}modes={\allowbreak}<modes>}.
    Special values: \ttt{default}, \ttt{all}: they select the modes listed in
    the \ttt{record} key of \ttt{@{\allowbreak}statistic}; all selects all of
    them, default selects the non-optional ones (i.e. excludes the ones that
    end in a question mark). Example values: \ttt{vector}, \ttt{count},
    \ttt{last}, \ttt{sum}, \ttt{mean}, \ttt{min}, \ttt{max}, \ttt{timeavg},
    \ttt{stats}, \ttt{histogram}. More than one values are accepted, separated
    by commas. Expressions are allowed. Items prefixed with
    \ttt{-{\allowbreak}} get removed from the list. Example:
    \ttt{**.{\allowbreak}queue\-Length.{\allowbreak}result-{\allowbreak}recording-{\allowbreak}modes={\allowbreak}default,{\allowbreak}-{\allowbreak}vector,{\allowbreak}+timeavg}
\item[resultdir-subdivision] = \textit{<bool>}, default: \ttt{false}\\
    \textit{Per-simulation-run setting.}\\
    Makes the results directory hierarchical by appending
    \ttt{\$\{{\allowbreak}iterationvarsd\}{\allowbreak}} to the value of the
    \ttt{result-{\allowbreak}dir} config option. This is useful if a parameter
    study produces a large number of runs (>10000), as many file managers and
    other tools (including the OMNeT++ IDE) struggle with directories
    containing that many files. An alternative to using this option is to
    include iteration variables directly in the value of the
    \ttt{result-{\allowbreak}dir} option.
\item[**.rng-\%] = \textit{<int>}\\
    \textit{Per-object setting for modules and channels.}\\
    Maps a module-local RNG to one of the global RNGs. Example:
    \ttt{**.{\allowbreak}gen.{\allowbreak}rng-{\allowbreak}1={\allowbreak}3}
    maps the local RNG 1 of modules matching \ttt{**.{\allowbreak}gen} to the
    global RNG 3. The value may be an expression, with the \ttt{index} and
    \ttt{ancestor\-Index()} operators being potentially very useful. The
    default is one-to-one mapping, i.e. RNG k of all modules refer to the
    global RNG k (\ttt{for
    {\allowbreak}k={\allowbreak}0..{\allowbreak}num-{\allowbreak}rngs-{\allowbreak}1}).\\
    Usage:
    \ttt{<module-{\allowbreak}full-{\allowbreak}path>.{\allowbreak}rng-{\allowbreak}<local-{\allowbreak}index>={\allowbreak}<global-{\allowbreak}index>}.
    Examples:
    \ttt{**.{\allowbreak}mac.{\allowbreak}rng-{\allowbreak}0={\allowbreak}1;
    {\allowbreak}**.{\allowbreak}source[*].{\allowbreak}rng-{\allowbreak}0={\allowbreak}index}
\item[rng-class] = \textit{<string>}, default: \ttt{omnetpp::{\allowbreak}cMersenne\-Twister}\\
    \textit{Per-simulation-run setting.}\\
    The random number generator class to be used. It can be
    \ttt{cMersenne\-Twister}, \ttt{cLCG32}, \ttt{cAkaroa\-RNG}, or you can use
    your own RNG class (it must be subclassed from \ttt{cRNG}).
\item[runnumber-width] = \textit{<int>}, default: \ttt{0}\\
    \textit{Per-simulation-run setting.}\\
    Setting a nonzero value will cause the \ttt{\$runnumber} variable to get
    padded with leading zeroes to the given length.
\item[**.scalar-recording] = \textit{<bool>}, default: \ttt{true}\\
    \textit{Per-object setting for scalar results.}\\
    Whether the matching output scalars and statistic objects should be
    recorded.\\
    Usage:
    \ttt{<module-{\allowbreak}full-{\allowbreak}path>.{\allowbreak}<scalar-{\allowbreak}name>.{\allowbreak}scalar-{\allowbreak}recording={\allowbreak}true/{\allowbreak}false}.
    To enable/disable individual recording modes for a @statistic (those added
    via the \ttt{record={\allowbreak}...{\allowbreak}} key of
    \ttt{@{\allowbreak}statistic} or the
    \ttt{**.{\allowbreak}result-{\allowbreak}recording-{\allowbreak}modes={\allowbreak}...{\allowbreak}}
    config option), use \ttt{<statistic-{\allowbreak}name>:{\allowbreak}<mode>}
    for \ttt{<scalar-{\allowbreak}name>}, and make sure the
    \ttt{@{\allowbreak}statistic} as a whole is not disabled with
    \ttt{**.{\allowbreak}<statistic-{\allowbreak}name>.{\allowbreak}statistic-{\allowbreak}recording={\allowbreak}false}.\\
    Example:
    \ttt{**.{\allowbreak}ping.{\allowbreak}round\-Trip\-Time:{\allowbreak}stddev.{\allowbreak}scalar-{\allowbreak}recording={\allowbreak}false}
\item[scheduler-class] = \textit{<string>}, default: \ttt{omnetpp::{\allowbreak}cSequential\-Scheduler}\\
    \textit{Per-simulation-run setting.}\\
    Part of the Envir plugin mechanism: selects the scheduler class. This
    plugin interface allows for implementing real-time, hardware-in-the-loop,
    distributed and distributed parallel simulation. The class has to implement
    the \ttt{cScheduler} interface.
\item[sectionbasedconfig-configreader-class] = \textit{<string>}\\
    \textit{Global setting (applies to all simulation runs).}\\
    When
    \ttt{configuration-{\allowbreak}class={\allowbreak}Section\-Based\-Configuration}:
    selects the configuration reader C++ class, which must subclass from
    \ttt{cConfiguration\-Reader}.
\item[seed-\%-lcg32] = \textit{<int>}\\
    \textit{Per-simulation-run setting.}\\
    When cLCG32 is selected as random number generator: seed for the kth RNG.
    (Substitute k for '\%' in the key.)
\item[seed-\%-mt] = \textit{<int>}\\
    \textit{Per-simulation-run setting.}\\
    When Mersenne Twister is selected as random number generator (default):
    seed for RNG number k. (Substitute k for '\%' in the key.)
\item[seed-\%-mt-p\%] = \textit{<int>}\\
    \textit{Per-simulation-run setting.}\\
    With parallel simulation: When Mersenne Twister is selected as random
    number generator (default): seed for RNG number k in partition number p.
    (Substitute k for the first '\%' in the key, and p for the second.)
\item[seed-set] = \textit{<int>}, default: \ttt{\$\{{\allowbreak}runnumber\}{\allowbreak}}\\
    \textit{Per-simulation-run setting.}\\
    Selects the kth set of automatic random number seeds for the simulation.
    Meaningful values include \ttt{\$\{{\allowbreak}repetition\}{\allowbreak}}
    which is the repeat loop counter (see \ttt{repeat} option), and
    \ttt{\$\{{\allowbreak}runnumber\}{\allowbreak}}.
\item[sim-time-limit] = \textit{<double>}, unit=\ttt{s}\\
    \textit{Per-simulation-run setting.}\\
    Stops the simulation when simulation time reaches the given limit. The
    default is no limit.
\item[simtime-resolution] = \textit{<custom>}, default: \ttt{ps}\\
    \textit{Global setting (applies to all simulation runs).}\\
    Sets the resolution for the 64-bit fixed-point simulation time
    representation. Accepted values are: second-or-smaller time units (\ttt{s},
    \ttt{ms}, \ttt{us}, \ttt{ns}, \ttt{ps}, \ttt{fs} or as), power-of-ten
    multiples of such units (e.g. 100ms), and base-10 scale exponents in the
    -18..0 range. The maximum representable simulation time depends on the
    resolution. The default is picosecond resolution, which offers a range of
    {\textasciitilde}110 days.
\item[simtime-scale] = \textit{<int>}, default: \ttt{-{\allowbreak}12}\\
    \textit{Global setting (applies to all simulation runs).}\\
    DEPRECATED in favor of simtime-resolution. Sets the scale exponent, and
    thus the resolution of time for the 64-bit fixed-point simulation time
    representation. Accepted values are -18..0; for example, -6 selects
    microsecond resolution. -12 means picosecond resolution, with a maximum
    simtime of {\textasciitilde}110 days.
\item[snapshot-file] = \textit{<filename>}, default: \ttt{\$\{{\allowbreak}resultdir\}{\allowbreak}/{\allowbreak}\$\{{\allowbreak}configname\}{\allowbreak}-{\allowbreak}\$\{{\allowbreak}iterationvarsf\}{\allowbreak}\#\$\{{\allowbreak}repetition\}{\allowbreak}.{\allowbreak}sna}\\
    \textit{Per-simulation-run setting.}\\
    Name of the snapshot file.
\item[snapshotmanager-class] = \textit{<string>}, default: \ttt{omnetpp::{\allowbreak}envir::{\allowbreak}File\-Snapshot\-Manager}\\
    \textit{Per-simulation-run setting.}\\
    Part of the Envir plugin mechanism: selects the class to handle streams to
    which snapshot() writes its output. The class has to implement the
    \ttt{cISnapshot\-Manager} interface.
\item[**.statistic-recording] = \textit{<bool>}, default: \ttt{true}\\
    \textit{Per-object setting for statistics (@statistic).}\\
    Whether the matching \ttt{@{\allowbreak}statistic} should be recorded. This
    option lets one completely disable all recording from a @statistic.
    Disabling a \ttt{@{\allowbreak}statistic} this way is more efficient than
    specifying
    \ttt{**.{\allowbreak}scalar-{\allowbreak}recording={\allowbreak}false} and
    \ttt{**.{\allowbreak}vector-{\allowbreak}recording={\allowbreak}false}
    together.\\
    Usage:
    \ttt{<module-{\allowbreak}full-{\allowbreak}path>.{\allowbreak}<statistic-{\allowbreak}name>.{\allowbreak}statistic-{\allowbreak}recording={\allowbreak}true/{\allowbreak}false}.\\
    Example:
    \ttt{**.{\allowbreak}ping.{\allowbreak}round\-Trip\-Time.{\allowbreak}statistic-{\allowbreak}recording={\allowbreak}false}
\item[total-stack] = \textit{<double>}, unit=\ttt{B}\\
    \textit{Global setting (applies to all simulation runs).}\\
    Specifies the maximum memory for \ttt{activity()} simple module stacks. You
    need to increase this value if you get a "Cannot allocate coroutine stack"
    error.
\item[**.typename] = \textit{<string>}\\
    \textit{Per-object setting for modules and channels.}\\
    Specifies type for submodules and channels declared with 'like <>'.
\item[user-interface] = \textit{<string>}\\
    \textit{Global setting (applies to all simulation runs).}\\
    Selects the user interface to be started. Known good values are Cmdenv and
    Qtenv. This option is normally left empty, as it is more convenient to
    specify the user interface via a command-line option or the IDE's Run and
    Debug dialogs. New user interfaces can be defined by subclassing
    \ttt{cRunnable\-Envir}.
\item[**.vector-buffer] = \textit{<double>}, unit=\ttt{B}, default: \ttt{1Mi\-B}\\
    \textit{Per-object setting for vector results.}\\
    For output vectors: the maximum per-vector buffer space used for storing
    values before writing them out as a block into the output vector file.
    There is also a total limit, see
    \ttt{output-{\allowbreak}vectors-{\allowbreak}memory-{\allowbreak}limit}.\\
    Usage:
    \ttt{<module-{\allowbreak}full-{\allowbreak}path>.{\allowbreak}<vector-{\allowbreak}name>.{\allowbreak}vector-{\allowbreak}buffer={\allowbreak}<amount>}.
\item[**.vector-record-empty] = \textit{<bool>}, default: \ttt{true}\\
    \textit{Per-object setting for vector results.}\\
    Whether output vectors that have no data recorded into them should appear
    in the result file. Turning off this option will delay writing the vector
    declaration into the result file until there is data recorded into the
    vector. Usage:
    \ttt{<module-{\allowbreak}full-{\allowbreak}path>.{\allowbreak}<vector-{\allowbreak}name>.{\allowbreak}vector-{\allowbreak}record-{\allowbreak}empty={\allowbreak}true/{\allowbreak}false}.\\
    Example:
    \ttt{**.{\allowbreak}ping.{\allowbreak}round\-Trip\-Time:{\allowbreak}vector.{\allowbreak}vector-{\allowbreak}record-{\allowbreak}empty={\allowbreak}false}
\item[**.vector-record-eventnumbers] = \textit{<bool>}, default: \ttt{true}\\
    \textit{Per-object setting for vector results.}\\
    Whether to record event numbers for an output vector. (Values and
    timestamps are always recorded.) Event numbers are needed by the Sequence
    Chart Tool, for example.\\
    Usage:
    \ttt{<module-{\allowbreak}full-{\allowbreak}path>.{\allowbreak}<vector-{\allowbreak}name>.{\allowbreak}vector-{\allowbreak}record-{\allowbreak}eventnumbers={\allowbreak}true/{\allowbreak}false}.\\
    Example:
    \ttt{**.{\allowbreak}ping.{\allowbreak}round\-Trip\-Time:{\allowbreak}vector.{\allowbreak}vector-{\allowbreak}record-{\allowbreak}eventnumbers={\allowbreak}false}
\item[**.vector-recording] = \textit{<bool>}, default: \ttt{true}\\
    \textit{Per-object setting for vector results.}\\
    Whether data written into an output vector should be recorded.\\
    Usage:
    \ttt{<module-{\allowbreak}full-{\allowbreak}path>.{\allowbreak}<vector-{\allowbreak}name>.{\allowbreak}vector-{\allowbreak}recording={\allowbreak}true/{\allowbreak}false}.
    To control vector recording from a \ttt{@{\allowbreak}statistic}, use
    \ttt{<statistic-{\allowbreak}name>:{\allowbreak}vector {\allowbreak}for
    {\allowbreak}<vector-{\allowbreak}name>}. Example:
    \ttt{**.{\allowbreak}ping.{\allowbreak}round\-Trip\-Time:{\allowbreak}vector.{\allowbreak}vector-{\allowbreak}recording={\allowbreak}false}
\item[**.vector-recording-intervals] = \textit{<custom>}\\
    \textit{Per-object setting for vector results.}\\
    Allows one to restrict recording of an output vector to one or more
    simulation time intervals. Usage:
    \ttt{<module-{\allowbreak}full-{\allowbreak}path>.{\allowbreak}<vector-{\allowbreak}name>.{\allowbreak}vector-{\allowbreak}recording-{\allowbreak}intervals={\allowbreak}<intervals>}.
    The syntax for \ttt{<intervals>} is:
    \ttt{[<from>]..{\allowbreak}[<to>],{\allowbreak}...{\allowbreak}} That is,
    both start and end of an interval are optional, and intervals are separated
    by comma.\\
    Example:
    \ttt{**.{\allowbreak}round\-Trip\-Time:{\allowbreak}vector.{\allowbreak}vector-{\allowbreak}recording-{\allowbreak}intervals={\allowbreak}..{\allowbreak}100,{\allowbreak}
    {\allowbreak}200..{\allowbreak}400,{\allowbreak}
    {\allowbreak}900..{\allowbreak}}
\item[warmup-period] = \textit{<double>}, unit=\ttt{s}\\
    \textit{Per-simulation-run setting.}\\
    Length of the initial warm-up period. When set, results belonging to the
    first x seconds of the simulation will not be recorded into output vectors,
    and will not be counted into output scalars (see option
    \ttt{**.{\allowbreak}result-{\allowbreak}recording-{\allowbreak}modes}).
    This option is useful for steady-state simulations. The default is 0s (no
    warmup period). Note that models that compute and record scalar results
    manually (via \ttt{record\-Scalar()}) will not automatically obey this
    setting.
\item[warnings] = \textit{<bool>}, default: \ttt{true}\\
    \textit{Per-simulation-run setting.}\\
    Enables warnings.
\end{description}
% @ENDFILE

\section{Predefined Variables}
\label{sec:config-options:predefined-variables}

Predefined variables that can be used in config values:

% @BEGINFILE tools/configvars.txt
\begin{description}
\item[\$\{runid\}] : \\
    A reasonably globally unique identifier for the run, produced by
    concatenating the configuration name, run number, date/time, etc.
\item[\$\{inifile\}] : \\
    Name of the (primary) inifile
\item[\$\{configname\}] : \\
    Name of the active configuration
\item[\$\{runnumber\}] : \\
    Sequence number of the current run within all runs in the active
    configuration
\item[\$\{network\}] : \\
    Value of the \ttt{network} configuration option
\item[\$\{experiment\}] : \\
    Value of the \ttt{experiment-{\allowbreak}label} configuration option
\item[\$\{measurement\}] : \\
    Value of the \ttt{measurement-{\allowbreak}label} configuration option
\item[\$\{replication\}] : \\
    Value of the \ttt{replication-{\allowbreak}label} configuration option
\item[\$\{processid\}] : \\
    PID of the simulation process
\item[\$\{datetime\}] : \\
    Date and time the simulation run was started
\item[\$\{datetimef\}] : \\
    Like \$\{datetime\}, but sanitized for use as part of a file name
\item[\$\{resultdir\}] : \\
    Value of the \ttt{result-{\allowbreak}dir} configuration option
\item[\$\{repetition\}] : \\
    The iteration number in \ttt{0..{\allowbreak}N-{\allowbreak}1}, where
    \ttt{N} is the value of the \ttt{repeat} configuration option
\item[\$\{seedset\}] : \\
    Value of the \ttt{seed-{\allowbreak}set} configuration option
\item[\$\{iterationvars\}] : \\
    Concatenation of all user-defined iteration variables in
    \ttt{name={\allowbreak}value} form
\item[\$\{iterationvarsf\}] : \\
    Like \$\{iterationvars\}, but sanitized for use as part of a file name
\item[\$\{iterationvarsd\}] : \\
    Like \$\{iterationvars\}, but for use as hierarchical folder name (it
    contains slashes where \$\{iterationvarsf\} has commas)
\end{description}
% @ENDFILE

%%% Local Variables:
%%% mode: latex
%%% TeX-master: "usman"
%%% End:
