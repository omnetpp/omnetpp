\appendixchapter{Python API for Chart Scripts}
\label{cha:chart-api}

This chapter describes the API of the Python modules available
for chart scripts. These modules are available in the Analysis Tool
in the IDE, in \fprog{opp\_charttool}, and may also be used
in standalone Python scripts.

Some conventional import aliases appear in code fragments throughout this
chapter, such as \ttt{np} for NumPy and \ttt{pd} for Pandas.

% --------
% The rest of this file is the output of the tools/extract_chart_api.py
% script (tools/chartapi.txt), pasted here manually, do not edit here!
% --------

%
% generated with extract_chart_api.py
%

\section{Module omnetpp.scave.results}
\label{cha:chart-api:omnetpp.scave.results}

\par This module lets the scripts that power the charts in the IDE query
any simulation results and metadata referenced by the .anf file they
are in, returned as as Pandas DataFrames in various formats.\par The \ttt{filter\_expressions} parameter in all functions has the same syntax.
It is always evaluated independently on every loaded result item or metadata entry, and its value
determines whether the given item or piece of metadata is included in the returned \ttt{DataFrame}.

\subsection{get\_serial()}
\label{cha:chart-api:omnetpp.scave.results:get-serial}

\begin{flushleft}
\ttt{get\_serial()}
\end{flushleft}

\par Returns an integer that is incremented every time the set of loaded results
change, typically as a result of the IDE loading, reloading or unloading
a scalar or vector result file. The serial can be used for invalidating
cached intermediate results when their input changes.

\subsection{get\_results()}
\label{cha:chart-api:omnetpp.scave.results:get-results}

\begin{flushleft}
\ttt{get\_results(filter\_expression="", row\_types=["runattr", "itervar", "config", "scalar", "vector", "statistic", "histogram", "param", "attr"], omit\_unused\_columns=True, start\_time=-inf, end\_time=inf)}
\end{flushleft}

\par Returns a filtered set of results and metadata in CSV-like format.
The items can be any type, even mixed together in a single \ttt{DataFrame}.
They are selected from the complete set of data referenced by the analysis file (\ttt{.anf}),
including only those for which the given \ttt{filter\_expression} evaluates to \ttt{True}.
\subsubsection{Parameters}
\label{sssec:num1}

\begin{itemize}
  \item \textbf{filter\_expression} \textit{(string)}: The filter expression to select the desired items. Example: \ttt{module ={\textasciitilde} "*host*" AND name ={\textasciitilde} "numPacket*"}
  \item \textbf{row\_types}: Optional. When given, filters the returned rows by type. Should be a unique list, containing any number of these strings:
\ttt{"runattr"}, \ttt{"itervar"}, \ttt{"config"}, \ttt{"scalar"}, \ttt{"vector"}, \ttt{"statistic"}, \ttt{"histogram"}, \ttt{"param"}, \ttt{"attr"}
  \item \textbf{omit\_unused\_columns} \textit{(bool)}: Optional. If \ttt{True}, all columns that would only contain \ttt{None} are removed from the returned DataFrame
  \item \textbf{start\_time}, \textbf{end\_time} \textit{(double)}: Optional time limits to trim the data of vector type results.
The unit is seconds, both the \ttt{vectime} and \ttt{vecvalue} arrays will be affected, the interval is left-closed, right-open.

\end{itemize}

\subsubsection{Columns of the returned DataFrame}
\label{sssec:num2}

\begin{itemize}
  \item \textbf{runID} \textit{(string)}:  Identifies the simulation run
  \item \textbf{type} \textit{(string)}: Row type, one of the following: scalar, vector, statistics, histogram, runattr, itervar, param, attr
  \item \textbf{module} \textit{(string)}: Hierarchical name (a.k.a. full path) of the module that recorded the result item
  \item \textbf{name} \textit{(string)}: Name of the result item (scalar, statistic, histogram or vector)
  \item \textbf{attrname} \textit{(string)}: Name of the run attribute or result item attribute (in the latter case, the module and name columns identify the result item the attribute belongs to)
  \item \textbf{attrvalue} \textit{(string)}: Value of run and result item attributes, iteration variables, saved ini param settings (runattr, attr, itervar, param)
  \item \textbf{value} \textit{(double or string)}:  Output scalar or attribute value
  \item \textbf{count}, \textbf{sumweights}, \textbf{mean}, \textbf{min}, \textbf{max}, \textbf{stddev} \textit{(double)}: Fields of the statistics or histogram
  \item \textbf{binedges}, \textbf{binvalues} \textit{(np.array)}: Histogram bin edges and bin values, as space-separated lists. \ttt{len(binedges)==len(binvalues)+1}
  \item \textbf{underflows}, \textbf{overflows} \textit{(double)}: Sum of weights (or counts) of underflown and overflown samples of histograms
  \item \textbf{vectime}, \textbf{vecvalue} \textit{(np.array)}: Output vector time and value arrays, as space-separated lists

\end{itemize}

\subsection{get\_runs()}
\label{cha:chart-api:omnetpp.scave.results:get-runs}

\begin{flushleft}
\ttt{get\_runs(filter\_expression="", include\_runattrs=False, include\_itervars=False, include\_param\_assignments=False, include\_config\_entries=False)}
\end{flushleft}

\par Returns a filtered list of runs, identified by their run ID.
\subsubsection{Parameters}
\label{sssec:num3}

\begin{itemize}
  \item \textbf{filter\_expression}: The filter expression to select the desired runs.
Example: \ttt{runattr:network ={\textasciitilde} "Aloha" AND config:Aloha.slotTime ={\textasciitilde} 0}
  \item \textbf{include\_runattrs}, \textbf{include\_itervars}, \textbf{include\_param\_assignments}, \textbf{include\_config\_entries} \textit{(bool)}:
Optional. When set to \ttt{True}, additional pieces of metadata about the run is appended to the result, pivoted into columns.

\end{itemize}

\subsubsection{Columns of the returned DataFrame}
\label{sssec:num4}

\begin{itemize}
  \item \textbf{runID} (string): Identifies the simulation run
  \item Additional metadata items (run attributes, iteration variables, etc.), as requested

\end{itemize}

\subsection{get\_runattrs()}
\label{cha:chart-api:omnetpp.scave.results:get-runattrs}

\begin{flushleft}
\ttt{get\_runattrs(filter\_expression="", include\_runattrs=False, include\_itervars=False, include\_param\_assignments=False, include\_config\_entries=False)}
\end{flushleft}

\par Returns a filtered list of run attributes.\par The set of run attributes is fixed: \ttt{configname}, \ttt{datetime}, \ttt{experiment},
\ttt{inifile}, \ttt{iterationvars}, \ttt{iterationvarsf}, \ttt{measurement}, \ttt{network},
\ttt{processid}, \ttt{repetition}, \ttt{replication}, \ttt{resultdir}, \ttt{runnumber}, \ttt{seedset}.
\subsubsection{Parameters}
\label{sssec:num5}

\begin{itemize}
  \item \textbf{filter\_expression}: The filter expression to select the desired run attributes.
Example: \ttt{name ={\textasciitilde} *date* AND config:Aloha.slotTime ={\textasciitilde} 0}
  \item \textbf{include\_runattrs}, \textbf{include\_itervars}, \textbf{include\_param\_assignments}, \textbf{include\_config\_entries} \textit{(bool)}:
Optional. When set to \ttt{True}, additional pieces of metadata about the run is appended to the result, pivoted into columns.

\end{itemize}

\subsubsection{Columns of the returned DataFrame}
\label{sssec:num6}

\begin{itemize}
  \item \textbf{runID} \textit{(string)}: Identifies the simulation run
  \item \textbf{name} \textit{(string)}: The name of the run attribute
  \item \textbf{value} \textit{(string)}: The value of the run attribue
  \item Additional metadata items (run attributes, iteration variables, etc.), as requested

\end{itemize}

\subsection{get\_itervars()}
\label{cha:chart-api:omnetpp.scave.results:get-itervars}

\begin{flushleft}
\ttt{get\_itervars(filter\_expression="", include\_runattrs=False, include\_itervars=False, include\_param\_assignments=False, include\_config\_entries=False, as\_numeric=False)}
\end{flushleft}

\par Returns a filtered list of iteration variables.
\subsubsection{Parameters}
\label{sssec:num7}

\begin{itemize}
  \item \textbf{filter\_expression} \textit{(string)}: The filter expression to select the desired iteration variables.
Example: \ttt{name ={\textasciitilde} iaMean AND config:Aloha.slotTime ={\textasciitilde} 0}
  \item \textbf{include\_runattrs}, \textbf{include\_itervars}, \textbf{include\_param\_assignments}, \textbf{include\_config\_entries} \textit{(bool)}:
Optional. When set to \ttt{True}, additional pieces of metadata about the run is appended to the result, pivoted into columns.
  \item \textbf{as\_numeric} \textit{(bool)}: Optional. When set to \ttt{True}, the returned values will be converted to \ttt{double}.
Non-numeric values will become \ttt{NaN}.

\end{itemize}

\subsubsection{Columns of the returned DataFrame}
\label{sssec:num8}

\begin{itemize}
  \item \textbf{runID} \textit{(string)}: Identifies the simulation run
  \item \textbf{name} \textit{(string)}: The name of the iteration variable
  \item \textbf{value} \textit{(string or double)}: The value of the iteration variable. Its type depends on the \ttt{as\_numeric} parameter.
  \item Additional metadata items (run attributes, iteration variables, etc.), as requested

\end{itemize}

\subsection{get\_scalars()}
\label{cha:chart-api:omnetpp.scave.results:get-scalars}

\begin{flushleft}
\ttt{get\_scalars(filter\_expression="", include\_attrs=False, include\_runattrs=False, include\_itervars=False, include\_param\_assignments=False, include\_config\_entries=False, merge\_module\_and\_name=False)}
\end{flushleft}

\par Returns a filtered list of scalar results.
\subsubsection{Parameters}
\label{sssec:num9}

\begin{itemize}
  \item \textbf{filter\_expression} \textit{(string)}: The filter expression to select the desired scalars.
Example: \ttt{name ={\textasciitilde} "channelUtilization*" AND runattr:replication ={\textasciitilde} "\#0"}
  \item \textbf{include\_attrs} \textit{(bool)}: Optional. When set to \ttt{True}, result attributes (like \ttt{unit}
or \ttt{source} for example) are appended to the DataFrame, pivoted into columns.
  \item \textbf{include\_runattrs}, \textbf{include\_itervars}, \textbf{include\_param\_assignments}, \textbf{include\_config\_entries} \textit{(bool)}:
Optional. When set to \ttt{True}, additional pieces of metadata about the run is appended to the DataFrame, pivoted into columns.
  \item \textbf{merge\_module\_and\_name} \textit{(bool)}: Optional. When set to \ttt{True}, the value in the \ttt{module} column
is prepended to the value in the \ttt{name} column, joined by a period, in every row.

\end{itemize}

\subsubsection{Columns of the returned DataFrame}
\label{sssec:num10}

\begin{itemize}
  \item \textbf{runID} \textit{(string)}: Identifies the simulation run
  \item \textbf{module} \textit{(string)}: Hierarchical name (a.k.a. full path) of the module that recorded the result item
  \item \textbf{name} \textit{(string)}: The name of the scalar
  \item \textbf{value} \textit{(double)}: The value of the scalar
  \item Additional metadata items (result attributes, run attributes, iteration variables, etc.), as requested

\end{itemize}

\subsection{get\_parameters()}
\label{cha:chart-api:omnetpp.scave.results:get-parameters}

\begin{flushleft}
\ttt{get\_parameters(filter\_expression="", include\_attrs=False, include\_runattrs=False, include\_itervars=False, include\_param\_assignments=False, include\_config\_entries=False, merge\_module\_and\_name=False, as\_numeric=False)}
\end{flushleft}

\par Returns a filtered list of parameters - actually computed values of individual \ttt{cPar} instances in the fully built network.\par Parameters are considered "pseudo-results", similar to scalars - except their values are strings. Even though they act
mostly as input to the actual simulation run, the actually assigned value of individual \ttt{cPar} instances is valuable information,
as it is the result of the network setup process. For example, even if a parameter is set up as an expression like \ttt{normal(3, 0.4)}
from \ttt{omnetpp.ini}, the returned DataFrame will contain the single concrete value picked for every instance of the parameter.
\subsubsection{Parameters}
\label{sssec:num11}

\begin{itemize}
  \item \textbf{filter\_expression} \textit{(string)}: The filter expression to select the desired parameters.
Example: \ttt{name ={\textasciitilde} "x" AND module ={\textasciitilde} Aloha.server}
  \item \textbf{include\_attrs} \textit{(bool)}: Optional. When set to \ttt{True}, result attributes (like \ttt{unit} for
example) are appended to the DataFrame, pivoted into columns.
  \item \textbf{include\_runattrs}, \textbf{include\_itervars}, \textbf{include\_param\_assignments}, \textbf{include\_config\_entries} \textit{(bool)}:
Optional. When set to \ttt{True}, additional pieces of metadata about the run is appended to the DataFrame, pivoted into columns.
  \item \textbf{merge\_module\_and\_name} \textit{(bool)}: Optional. When set to \ttt{True}, the value in the \ttt{module} column
is prepended to the value in the \ttt{name} column, joined by a period, in every row.
  \item \textbf{as\_numeric} \textit{(bool)}: Optional. When set to \ttt{True}, the returned values will be converted to \ttt{double}.
Non-numeric values will become \ttt{NaN}.

\end{itemize}

\subsubsection{Columns of the returned DataFrame}
\label{sssec:num12}

\begin{itemize}
  \item \textbf{runID} \textit{(string)}: Identifies the simulation run
  \item \textbf{module} \textit{(string)}: Hierarchical name (a.k.a. full path) of the module that recorded the result item
  \item \textbf{name} \textit{(string)}: The name of the parameter
  \item \textbf{value} \textit{(string or double)}: The value of the parameter. Its type depends on the \ttt{as\_numeric} parameter.
  \item Additional metadata items (result attributes, run attributes, iteration variables, etc.), as requested

\end{itemize}

\subsection{get\_vectors()}
\label{cha:chart-api:omnetpp.scave.results:get-vectors}

\begin{flushleft}
\ttt{get\_vectors(filter\_expression="", include\_attrs=False, include\_runattrs=False, include\_itervars=False, include\_param\_assignments=False, include\_config\_entries=False, merge\_module\_and\_name=False, start\_time=-inf, end\_time=inf)}
\end{flushleft}

\par Returns a filtered list of vector results.
\subsubsection{Parameters}
\label{sssec:num13}

\begin{itemize}
  \item \textbf{filter\_expression} \textit{(string)}: The filter expression to select the desired vectors.
Example: \ttt{name ={\textasciitilde} "radioState*" AND runattr:replication ={\textasciitilde} "\#0"}
  \item \textbf{include\_attrs} \textit{(bool)}: Optional. When set to \ttt{True}, result attributes (like \ttt{unit}
or \ttt{source} for example) are appended to the DataFrame, pivoted into columns.
  \item \textbf{include\_runattrs}, \textbf{include\_itervars}, \textbf{include\_param\_assignments}, \textbf{include\_config\_entries} \textit{(bool)}:
Optional. When set to \ttt{True}, additional pieces of metadata about the run is appended to the DataFrame, pivoted into columns.
  \item \textbf{merge\_module\_and\_name} \textit{(bool)}: Optional. When set to \ttt{True}, the value in the \ttt{module} column
is prepended to the value in the \ttt{name} column, joined by a period, in every row.
  \item \textbf{start\_time}, \textbf{end\_time} \textit{(double)}: Optional time limits to trim the data of vector type results.
The unit is seconds, both the \ttt{vectime} and \ttt{vecvalue} arrays will be affected, the interval is left-closed, right-open.

\end{itemize}

\subsubsection{Columns of the returned DataFrame}
\label{sssec:num14}

\begin{itemize}
  \item \textbf{runID} \textit{(string)}: Identifies the simulation run
  \item \textbf{module} \textit{(string)}: Hierarchical name (a.k.a. full path) of the module that recorded the result item
  \item \textbf{name} \textit{(string)}: The name of the vector
  \item \textbf{vectime}, \textbf{vecvalue} \textit{(np.array)}: The simulation times and the corresponding values in the vector
  \item Additional metadata items (result attributes, run attributes, iteration variables, etc.), as requested

\end{itemize}

\subsection{get\_statistics()}
\label{cha:chart-api:omnetpp.scave.results:get-statistics}

\begin{flushleft}
\ttt{get\_statistics(filter\_expression="", include\_attrs=False, include\_runattrs=False, include\_itervars=False, include\_param\_assignments=False, include\_config\_entries=False, merge\_module\_and\_name=False)}
\end{flushleft}

\par Returns a filtered list of statistics results.
\subsubsection{Parameters}
\label{sssec:num15}

\begin{itemize}
  \item \textbf{filter\_expression} \textit{(string)}: The filter expression to select the desired statistics.
Example: \ttt{name ={\textasciitilde} "collisionLength:stat" AND itervar:iaMean ={\textasciitilde} "5"}
  \item \textbf{include\_attrs} \textit{(bool)}: Optional. When set to \ttt{True}, result attributes (like \ttt{unit}
or \ttt{source} for example) are appended to the DataFrame, pivoted into columns.
  \item \textbf{include\_runattrs}, \textbf{include\_itervars}, \textbf{include\_param\_assignments}, \textbf{include\_config\_entries} \textit{(bool)}:
Optional. When set to \ttt{True}, additional pieces of metadata about the run is appended to the DataFrame, pivoted into columns.
  \item \textbf{merge\_module\_and\_name} \textit{(bool)}: Optional. When set to \ttt{True}, the value in the \ttt{module} column
is prepended to the value in the \ttt{name} column, joined by a period, in every row.

\end{itemize}

\subsubsection{Columns of the returned DataFrame}
\label{sssec:num16}

\begin{itemize}
  \item \textbf{runID} \textit{(string)}: Identifies the simulation run
  \item \textbf{module} \textit{(string)}: Hierarchical name (a.k.a. full path) of the module that recorded the result item
  \item \textbf{name} \textit{(string)}: The name of the vector
  \item \textbf{count}, \textbf{sumweights}, \textbf{mean}, \textbf{stddev}, \textbf{min}, \textbf{max} \textit{(double)}: The characteristic mathematical properties of the statistics result
  \item Additional metadata items (result attributes, run attributes, iteration variables, etc.), as requested

\end{itemize}

\subsection{get\_histograms()}
\label{cha:chart-api:omnetpp.scave.results:get-histograms}

\begin{flushleft}
\ttt{get\_histograms(filter\_expression="", include\_attrs=False, include\_runattrs=False, include\_itervars=False, include\_param\_assignments=False, include\_config\_entries=False, merge\_module\_and\_name=False, include\_statistics\_fields=False)}
\end{flushleft}

\par Returns a filtered list of histogram results.
\subsubsection{Parameters}
\label{sssec:num17}

\begin{itemize}
  \item \textbf{filter\_expression} \textit{(string)}: The filter expression to select the desired histogram.
Example: \ttt{name ={\textasciitilde} "collisionMultiplicity:histogram" AND itervar:iaMean ={\textasciitilde} "2"}
  \item \textbf{include\_attrs} \textit{(bool)}: Optional. When set to \ttt{True}, result attributes (like \ttt{unit}
or \ttt{source} for example) are appended to the DataFrame, pivoted into columns.
  \item \textbf{include\_runattrs}, \textbf{include\_itervars}, \textbf{include\_param\_assignments}, \textbf{include\_config\_entries} \textit{(bool)}:
Optional. When set to \ttt{True}, additional pieces of metadata about the run is appended to the DataFrame, pivoted into columns.
  \item \textbf{merge\_module\_and\_name} \textit{(bool)}: Optional. When set to \ttt{True}, the value in the \ttt{module} column
is prepended to the value in the \ttt{name} column, joined by a period, in every row.

\end{itemize}

\subsubsection{Columns of the returned DataFrame}
\label{sssec:num18}

\begin{itemize}
  \item \textbf{runID} \textit{(string)}: Identifies the simulation run
  \item \textbf{module} \textit{(string)}: Hierarchical name (a.k.a. full path) of the module that recorded the result item
  \item \textbf{name} \textit{(string)}: The name of the vector
  \item \textbf{count}, \textbf{sumweights}, \textbf{mean}, \textbf{stddev}, \textbf{min}, \textbf{max} \textit{(double)}: The characteristic mathematical properties of the histogram
  \item \textbf{binedges}, \textbf{binvalues} \textit{(np.array)}: The histogram edge locations and the weighted sum of the collected samples in each bin. \ttt{len(binedges) == len(binvalues) + 1}
  \item \textbf{underflows}, \textbf{overflows} \textit{(double)}: The weighted sum of the samples that fell outside of the histogram bin range in the two directions
  \item Additional metadata items (result attributes, run attributes, iteration variables, etc.), as requested

\end{itemize}

\subsection{get\_config\_entries()}
\label{cha:chart-api:omnetpp.scave.results:get-config-entries}

\begin{flushleft}
\ttt{get\_config\_entries(filter\_expression, include\_runattrs=False, include\_itervars=False, include\_param\_assignments=False, include\_config\_entries=False)}
\end{flushleft}

\par Returns a filtered list of config entries. That is: parameter assignment patterns; and global and per-object config options.
\subsubsection{Parameters}
\label{sssec:num19}

\begin{itemize}
  \item \textbf{filter\_expression} \textit{(string)}: The filter expression to select the desired config entries.
Example: \ttt{name ={\textasciitilde} sim-time-limit AND itervar:numHosts ={\textasciitilde} 10}
  \item \textbf{include\_runattrs}, \textbf{include\_itervars}, \textbf{include\_param\_assignments}, \textbf{include\_config\_entries} \textit{(bool)}:
Optional. When set to \ttt{True}, additional pieces of metadata about the run is appended to the result, pivoted into columns.

\end{itemize}

\subsubsection{Columns of the returned DataFrame}
\label{sssec:num20}

\begin{itemize}
  \item \textbf{runID} \textit{(string)}: Identifies the simulation run
  \item \textbf{name} \textit{(string)}: The name of the config entry
  \item \textbf{value} \textit{(string or double)}: The value of the config entry
  \item Additional metadata items (run attributes, iteration variables, etc.), as requested

\end{itemize}

\section{Module omnetpp.scave.chart}
\label{cha:chart-api:omnetpp.scave.chart}

\par This module provides functions to access the properties of the chart object.

\subsection{get\_properties()}
\label{cha:chart-api:omnetpp.scave.chart:get-properties}

\begin{flushleft}
\ttt{get\_properties()}
\end{flushleft}

\par Returns the currently set properties of the chart as a \ttt{dict}
whose keys and values are both all strings.

\subsection{get\_property()}
\label{cha:chart-api:omnetpp.scave.chart:get-property}

\begin{flushleft}
\ttt{get\_property(key)}
\end{flushleft}

\par Returns the value of a single property of the chart, or \ttt{None} if there is
no property with the given name (key) set on the chart.

\subsection{get\_name()}
\label{cha:chart-api:omnetpp.scave.chart:get-name}

\begin{flushleft}
\ttt{get\_name()}
\end{flushleft}

\par Returns the name of the chart as a string.

\subsection{get\_chart\_type()}
\label{cha:chart-api:omnetpp.scave.chart:get-chart-type}

\begin{flushleft}
\ttt{get\_chart\_type()}
\end{flushleft}

\par Returns the chart type, one of the strings "bar"/"histogram"/"line"/"matplotlib"

\subsection{is\_native\_chart()}
\label{cha:chart-api:omnetpp.scave.chart:is-native-chart}

\begin{flushleft}
\ttt{is\_native\_chart()}
\end{flushleft}

\par Returns True if this chart uses the IDE's built-in plotting widgets.

\subsection{set\_suggested\_chart\_name()}
\label{cha:chart-api:omnetpp.scave.chart:set-suggested-chart-name}

\begin{flushleft}
\ttt{set\_suggested\_chart\_name(name)}
\end{flushleft}

\par Sets a proposed name for the chart. The IDE may offer this name to the user
when saving the chart.

\section{Module omnetpp.scave.utils}
\label{cha:chart-api:omnetpp.scave.utils}

\par A collection of utility function for data manipulation and plotting, built
on top of Pandas data frames and the \ttt{chart} and \ttt{plot} packages from \ttt{omnetpp.scave}.
Functions in this module have been written largely to the needs of the
chart templates that ship with the IDE.

\subsection{confidence\_interval()}
\label{cha:chart-api:omnetpp.scave.utils:confidence-interval}

\begin{flushleft}
\ttt{confidence\_interval(alpha, data)}
\end{flushleft}

\par Returns the half-length of the confidence interval of the mean of \ttt{data}, assuming
normal distribution, for the given confidence level \ttt{alpha}.
\ttt{alpha} must be in the [0..1] range.

\subsection{split()}
\label{cha:chart-api:omnetpp.scave.utils:split}

\begin{flushleft}
\ttt{split(s, sep=",")}
\end{flushleft}

\par Split a string with the given separator (by default with comma), trim
the surrounding whitespace from the items, and return the result as a
list. Return an empty list for an empty or all-whitespace input string.

\subsection{extract\_label\_columns()}
\label{cha:chart-api:omnetpp.scave.utils:extract-label-columns}

\begin{flushleft}
\ttt{extract\_label\_columns(df, preferred\_legend\_column="title")}
\end{flushleft}

\par Utility function to make a reasonable guess as to which column of
the given DataFrame is most suitable to act as a chart title and
which ones can be used as legend labels.\par Ideally a "title column" should be one in which all lines have the same
value, and can be reasonably used as a title. Some often used candidates
are: \ttt{title}, \ttt{name}, and \ttt{module}.\par Label columns should be a minimal set of columns whose corresponding
value tuples uniquely identify every line in the DataFrame. These will
primarily be iteration variables and run attributes.
\subsubsection{Returns:}
\label{sssec:num21}

\par A pair of a string and a list; the first value is the name of the
"title" column, and the second one is a list of pairs, each
containing the index and the name of a "label" column.\par Example: \ttt{('title', [(8, 'numHosts'), (7, 'iaMean')])}

\subsection{make\_legend\_label()}
\label{cha:chart-api:omnetpp.scave.utils:make-legend-label}

\begin{flushleft}
\ttt{make\_legend\_label(legend\_cols, row)}
\end{flushleft}

\par Produces a reasonably good label text (to be used in a chart legend) for a result row from
a DataFrame, given a list of selected columns as returned by \ttt{extract\_label\_columns()}.

\subsection{make\_chart\_title()}
\label{cha:chart-api:omnetpp.scave.utils:make-chart-title}

\begin{flushleft}
\ttt{make\_chart\_title(df, title\_col, legend\_cols)}
\end{flushleft}

\par Produces a reasonably good chart title text from a result DataFrame, given a selected "title"
column, and a list of selected "legend" columns as returned by \ttt{extract\_label\_columns()}.

\subsection{pick\_two\_columns()}
\label{cha:chart-api:omnetpp.scave.utils:pick-two-columns}

\begin{flushleft}
\ttt{pick\_two\_columns(df)}
\end{flushleft}



\subsection{assert\_columns\_exist()}
\label{cha:chart-api:omnetpp.scave.utils:assert-columns-exist}

\begin{flushleft}
\ttt{assert\_columns\_exist(df, cols, message="Expected column missing from DataFrame")}
\end{flushleft}



\subsection{parse\_rcparams()}
\label{cha:chart-api:omnetpp.scave.utils:parse-rcparams}

\begin{flushleft}
\ttt{parse\_rcparams(rc\_content)}
\end{flushleft}



\subsection{make\_fancy\_xticklabels()}
\label{cha:chart-api:omnetpp.scave.utils:make-fancy-xticklabels}

\begin{flushleft}
\ttt{make\_fancy\_xticklabels(ax)}
\end{flushleft}



\subsection{make\_scroll\_navigable()}
\label{cha:chart-api:omnetpp.scave.utils:make-scroll-navigable}

\begin{flushleft}
\ttt{make\_scroll\_navigable(figure)}
\end{flushleft}



\subsection{customized\_box\_plot()}
\label{cha:chart-api:omnetpp.scave.utils:customized-box-plot}

\begin{flushleft}
\ttt{customized\_box\_plot(percentiles, axes, redraw=True, *args, **kwargs)}
\end{flushleft}

\par Generates a customized boxplot based on the given percentile values

\subsection{interpolationmode\_to\_drawstyle()}
\label{cha:chart-api:omnetpp.scave.utils:interpolationmode-to-drawstyle}

\begin{flushleft}
\ttt{interpolationmode\_to\_drawstyle(interpolationmode, hasenum)}
\end{flushleft}



\subsection{get\_names\_for\_title()}
\label{cha:chart-api:omnetpp.scave.utils:get-names-for-title}

\begin{flushleft}
\ttt{get\_names\_for\_title(df, props)}
\end{flushleft}



\subsection{set\_plot\_title()}
\label{cha:chart-api:omnetpp.scave.utils:set-plot-title}

\begin{flushleft}
\ttt{set\_plot\_title(title, suggested\_chart\_name=None)}
\end{flushleft}



\subsection{plot\_bars()}
\label{cha:chart-api:omnetpp.scave.utils:plot-bars}

\begin{flushleft}
\ttt{plot\_bars(df, props, names=None, errors\_df=None)}
\end{flushleft}



\subsection{plot\_vectors()}
\label{cha:chart-api:omnetpp.scave.utils:plot-vectors}

\begin{flushleft}
\ttt{plot\_vectors(df, props)}
\end{flushleft}



\subsection{plot\_histograms()}
\label{cha:chart-api:omnetpp.scave.utils:plot-histograms}

\begin{flushleft}
\ttt{plot\_histograms(df, props)}
\end{flushleft}



\subsection{preconfigure\_plot()}
\label{cha:chart-api:omnetpp.scave.utils:preconfigure-plot}

\begin{flushleft}
\ttt{preconfigure\_plot(props)}
\end{flushleft}



\subsection{postconfigure\_plot()}
\label{cha:chart-api:omnetpp.scave.utils:postconfigure-plot}

\begin{flushleft}
\ttt{postconfigure\_plot(props)}
\end{flushleft}



\subsection{legend()}
\label{cha:chart-api:omnetpp.scave.utils:legend}

\begin{flushleft}
\ttt{legend(*args, **kwargs)}
\end{flushleft}



\subsection{export\_image\_if\_needed()}
\label{cha:chart-api:omnetpp.scave.utils:export-image-if-needed}

\begin{flushleft}
\ttt{export\_image\_if\_needed(props)}
\end{flushleft}



\subsection{export\_data\_if\_needed()}
\label{cha:chart-api:omnetpp.scave.utils:export-data-if-needed}

\begin{flushleft}
\ttt{export\_data\_if\_needed(df, props)}
\end{flushleft}



\section{Module omnetpp.scave.plot}
\label{cha:chart-api:omnetpp.scave.plot}

\par This module is the interface for displaying data using the IDE's native
(non-Matplotlib) plotting widgets. The API is intentionally very close to
\ttt{matplotlib.pyplot}: most functions and the parameters they accept are a
subset of \ttt{pyplot}'s. The promise is that a chart script written with this
API is very easy to switch over to Matplotlib: usually, just importing
\ttt{matplotlib.pyplot} as \ttt{plot} (instead of importing this package) is
sufficient. Also the other way round: if a chart script uses only this
subset of \ttt{pyplot}'s API, it can be easily switched over to plot with the
IDE's native widgets.\par When the API is used outside the context of a native plotting widget
(such as during the run of \ttt{opp\_charttool}, or in IDE during image export),
the functions are emulated with Matplotlib.

\subsection{is\_native\_plot()}
\label{cha:chart-api:omnetpp.scave.plot:is-native-plot}

\begin{flushleft}
\ttt{is\_native\_plot()}
\end{flushleft}

\par Returns True if the script is running in the context of a native plotting
widget, and False otherwise.

\subsection{plot\_bars()}
\label{cha:chart-api:omnetpp.scave.plot:plot-bars}

\begin{flushleft}
\ttt{plot\_bars(df)}
\end{flushleft}

\par TODO

\subsection{plot\_lines()}
\label{cha:chart-api:omnetpp.scave.plot:plot-lines}

\begin{flushleft}
\ttt{plot\_lines(df)}
\end{flushleft}

\par TODO

\subsection{plot\_histograms()}
\label{cha:chart-api:omnetpp.scave.plot:plot-histograms}

\begin{flushleft}
\ttt{plot\_histograms(df)}
\end{flushleft}

\par TODO

\subsection{plot()}
\label{cha:chart-api:omnetpp.scave.plot:plot}

\begin{flushleft}
\ttt{plot(xs, ys, key=None, label=None, drawstyle=None, linestyle=None, linewidth=None, color=None, marker=None, markersize=None)}
\end{flushleft}

\par TODO

\subsection{hist()}
\label{cha:chart-api:omnetpp.scave.plot:hist}

\begin{flushleft}
\ttt{hist(x, bins, density=None, weights=None, cumulative=False, bottom=None, histtype="stepfilled", color=None, label=None, linewidth=None, underflows=0.0, overflows=0.0, minvalue=nan, maxvalue=nan)}
\end{flushleft}

\par TODO

\subsection{bar()}
\label{cha:chart-api:omnetpp.scave.plot:bar}

\begin{flushleft}
\ttt{bar(x, height, width=0.8, label=None, color=None, edgecolor=None)}
\end{flushleft}

\par Make a bar plot. This function adds one series to the bar plot; make
multiple calls to add multiple series.\par The bars are positioned at x with the given alignment. Their dimensions
are given by width and height. The vertical baseline is bottom (default 0).\par Each of x, height, width, and bottom may either be a scalar applying to
all bars, or it may be a sequence of length N providing a separate value
for each bar.
\subsubsection{Parameters}
\label{sssec:num22}

\begin{itemize}
  \item \textbf{x} \textit{(sequence of scalars)}: The x coordinates of the bars.
  \item \textbf{height} \textit{(scalar or sequence of scalars)}: The height(s) of the bars.
  \item \textbf{width} \textit{(scalar or array-like)}: The width(s) of the bars.
  \item \textbf{label} \textit{(string)}: The label of the series the bars represent .
  \item \textbf{color} \textit{(string)}: The fill color of the bars.
  \item \textbf{edgecolor} \textit{(string)}: The edge color of the bars.

\end{itemize}
\par The native plot implementation has the following restrictions:\begin{itemize}
  \item widths are automatic (parameter is ignored)
  \item x coordinates are automatic (values are ignored)
  \item height must be a sequence (cannot be a constant)
  \item in multiple calls to bar(), the lengths of the height sequence must be
equal (i.e. all series must have the same number of values)
  \item default color is grey (Matplotlib assigns a different color to each series)

\end{itemize}

\subsection{set\_property()}
\label{cha:chart-api:omnetpp.scave.plot:set-property}

\begin{flushleft}
\ttt{set\_property(key, value)}
\end{flushleft}

\par Sets one property of the native plot widget to the given value. When invoked
outside the contex of a native plot widget, the function does nothing.
\subsubsection{Parameters}
\label{sssec:num23}

\begin{itemize}
  \item \textbf{key} \textit{(string)}: Name of the property.
  \item \textbf{value} \textit{(string)}: The value to set. If any other type than string is passed in, it will be converted to string.

\end{itemize}

\subsection{set\_properties()}
\label{cha:chart-api:omnetpp.scave.plot:set-properties}

\begin{flushleft}
\ttt{set\_properties(props)}
\end{flushleft}

\par Sets several properties of the native plot widget. It is functionally equivalent to
repeatedly calling \ttt{set\_property} with the entries of the \ttt{props} dictionary.
When invoked outside the contex of a native plot widget (TODO?), the function does nothing.
\subsubsection{Parameters}
\label{sssec:num24}

\begin{itemize}
  \item \textbf{props} \textit{(dict)}: The properties to set.

\end{itemize}

\subsection{get\_supported\_property\_keys()}
\label{cha:chart-api:omnetpp.scave.plot:get-supported-property-keys}

\begin{flushleft}
\ttt{get\_supported\_property\_keys()}
\end{flushleft}

\par Returns the list of property names that the native plot widget supports, such as
'Plot.Title', 'X.Axis.Max' and 'Legend.Display', among many others.\par Note: This method has no equivalent in \ttt{pyplot}. When the script runs outside the IDE  (TODO?),
the method returns an empty list.

\subsection{set\_warning()}
\label{cha:chart-api:omnetpp.scave.plot:set-warning}

\begin{flushleft}
\ttt{set\_warning(warning)}
\end{flushleft}

\par Displays the given warning text in the plot.
\subsubsection{Parameters}
\label{sssec:num25}

\begin{itemize}
  \item \textbf{warning} \textit{(string)}: The warning string.

\end{itemize}

\subsection{title()}
\label{cha:chart-api:omnetpp.scave.plot:title}

\begin{flushleft}
\ttt{title(label)}
\end{flushleft}

\par Sets the plot title.
\subsubsection{Parameters}
\label{sssec:num26}

\begin{itemize}
  \item \textbf{label} \textit{(string)}: The plot title.

\end{itemize}

\subsection{xlabel()}
\label{cha:chart-api:omnetpp.scave.plot:xlabel}

\begin{flushleft}
\ttt{xlabel(xlabel)}
\end{flushleft}

\par Sets the label of the X axis.
\subsubsection{Parameters}
\label{sssec:num27}

\begin{itemize}
  \item \textbf{xlabel} \textit{(string)}: The label string.

\end{itemize}

\subsection{ylabel()}
\label{cha:chart-api:omnetpp.scave.plot:ylabel}

\begin{flushleft}
\ttt{ylabel(ylabel)}
\end{flushleft}

\par Sets the label of the Y axis.
\subsubsection{Parameters}
\label{sssec:num28}

\begin{itemize}
  \item \textbf{ylabel} \textit{(string)}: The label string.

\end{itemize}

\subsection{xlim()}
\label{cha:chart-api:omnetpp.scave.plot:xlim}

\begin{flushleft}
\ttt{xlim(left=None, right=None)}
\end{flushleft}

\par Sets the limits of the X axis.
\subsubsection{Parameters}
\label{sssec:num29}

\begin{itemize}
  \item \textbf{left} \textit{(float)}: The left xlim in data coordinates. Passing None leaves the limit unchanged.
  \item \textbf{right} \textit{(float)}: The right xlim in data coordinates. Passing None leaves the limit unchanged.

\end{itemize}

\subsection{ylim()}
\label{cha:chart-api:omnetpp.scave.plot:ylim}

\begin{flushleft}
\ttt{ylim(bottom=None, top=None)}
\end{flushleft}

\par Sets the limits of the Y axis.
\subsubsection{Parameters}
\label{sssec:num30}

\begin{itemize}
  \item \textbf{bottom} \textit{(float)}: The bottom ylim in data coordinates. Passing None leaves the limit unchanged.
  \item \textbf{top} \textit{(float)}: The top ylim in data coordinates. Passing None leaves the limit unchanged.

\end{itemize}

\subsection{xscale()}
\label{cha:chart-api:omnetpp.scave.plot:xscale}

\begin{flushleft}
\ttt{xscale(value)}
\end{flushleft}

\par Sets the scale of the X axis. Possible values are 'linear' and 'log'.
\subsubsection{Parameters}
\label{sssec:num31}

\begin{itemize}
  \item \textbf{value} \textit{(string)}: The scale. Possible values are 'linear' and 'log'.

\end{itemize}

\subsection{yscale()}
\label{cha:chart-api:omnetpp.scave.plot:yscale}

\begin{flushleft}
\ttt{yscale(value)}
\end{flushleft}

\par Sets the scale of the Y axis.
\subsubsection{Parameters}
\label{sssec:num32}

\begin{itemize}
  \item \textbf{value} \textit{(string)}: The scale. Possible values are 'linear' and 'log'.

\end{itemize}

\subsection{xticks()}
\label{cha:chart-api:omnetpp.scave.plot:xticks}

\begin{flushleft}
\ttt{xticks(ticks=None, labels=None, rotation=0)}
\end{flushleft}

\par Sets the current tick locations and labels of the x-axis.
\subsubsection{Parameters}
\label{sssec:num33}

\begin{itemize}
  \item \textbf{ticks} \textit{(array\_like)}: A list of positions at which ticks should be placed. You can pass an empty list to disable xticks.
  \item \textbf{labels} \textit{(array\_like)}: A list of explicit labels to place at the given locs.
  \item \textbf{rotation} \textit{(float)}: Label rotation in degrees.

\end{itemize}

\subsection{grid()}
\label{cha:chart-api:omnetpp.scave.plot:grid}

\begin{flushleft}
\ttt{grid(b=True, which="major")}
\end{flushleft}

\par Configure the grid lines.
\subsubsection{Parameters}
\label{sssec:num34}

\begin{itemize}
  \item \textbf{b} \textit{(bool or \ttt{None})}: Whether to show the grid lines.
  \item \textbf{which} \textit{('major', 'minor' or 'both')}: The grid lines to apply the changes on.

\end{itemize}

\subsection{legend()}
\label{cha:chart-api:omnetpp.scave.plot:legend}

\begin{flushleft}
\ttt{legend(show=None, frameon=None, loc=None)}
\end{flushleft}

\par Place a legend on the axes.
\subsubsection{Parameters}
\label{sssec:num35}

\begin{itemize}
  \item \textbf{show} \textit{(bool or \ttt{None})}: Whether to show the legend. TODO does pyplot have this?
  \item \textbf{frameon} \textit{(bool or \ttt{None})}: Control whether the legend should be drawn on a patch (frame).
  Default is \ttt{None}, which will take the value from the resource file.
  \item \textbf{loc} \textit{(string or \ttt{None})}: The location of the legend. Possible values are
  'best', 'upper right', 'upper left', 'lower left', 'lower right', 'right',
  'center left', 'center right', 'lower center', 'upper center', 'center'
  (these are the values supported by Matplotlib), plus additionally
  'outside top left', 'outside top center', 'outside top right',
  'outside bottom left', 'outside bottom center', 'outside bottom right',
  'outside left top', 'outside left center', 'outside left bottom',
  'outside right top', 'outside right center', 'outside right bottom'.

\end{itemize}

